%%This is a very basic article template.
%%There is just one section and two subsections.
\documentclass[a4paper]{article}

%definitions for Formelsammlung

\usepackage[left=1.5cm,right=1.5cm,top=2.5cm,bottom=2cm,landscape]{geometry} 
\usepackage{multicol}
\usepackage[ngerman]{babel}
\usepackage{tabularx}
\usepackage{mathpazo}
\usepackage{mathtools}
\usepackage{amsmath}  
\usepackage{setspace} 
\usepackage{commath}
\usepackage[utf8]{inputenc}
%\usepackage[ansinew]{inputenc}  
\usepackage[T1]{fontenc}
\usepackage{lmodern} 
\usepackage{hyperref}
\usepackage{bigints}
\usepackage{array}
\usepackage{xcolor}
\usepackage{layouts}
\usepackage{siunitx}
\usepackage{wrapfig}
\usepackage{multirow,bigstrut}
\usepackage{trfsigns}
\usepackage{amssymb} 
\usepackage{fancyhdr}
\usepackage{datetime}
\usepackage{pgfplots}
\usepgfplotslibrary{fillbetween}
\usepackage{listings}
\usepackage{mathrsfs}
\usepackage{booktabs}
%\usepackage{mathabx}

\DeclareMathOperator\arctanh{arctanh}
\DeclareMathOperator\arsinh{arsinh} 
\DeclareMathOperator\arcosh{arcosh}
\DeclareMathOperator\artanh{artanh}
\DeclareMathOperator\arcoth{arcoth} 
\DeclareMathOperator\sinc{sinc} 
\DeclareMathOperator\sgn{sgn} 
\DeclareMathOperator\LPF{LPF} 
\DeclareMathOperator\Q{Q} 
\DeclareMathOperator\erf{erf} 


%colorCodes
\definecolor{listinggray}{gray}{0.9}
\definecolor{lbcolor}{rgb}{0.95,0.95,0.95}
\definecolor{lightGray}{gray}{0.1}

\definecolor{cOrange}{HTML}{996633}
\definecolor{clOrange}{HTML}{DBB48D}
\definecolor{cBlue}{HTML}{336699}
\definecolor{clBlue}{HTML}{A0BCD8}
\definecolor{cGreen}{HTML}{339966}
\definecolor{clGreen}{HTML}{94D4B4}
\definecolor{cRed}{HTML}{993333}
\definecolor{clRed}{HTML}{D0B0B0}
\definecolor{cGray}{gray}{0.4}
\definecolor{clGray}{gray}{0.96}



\setlength{\parindent}{0pt}
%\DeclareMathOperator\arctanh{arccot}
\newcolumntype{L}[1]{>{\raggedright\let\newline\\\arraybackslash\hspace{0pt}}m{#1}}
\newcolumntype{C}[1]{>{\centering\let\newline\\\arraybackslash\hspace{0pt}}m{#1}}
\newcolumntype{R}[1]{>{\raggedleft\let\newline\\\arraybackslash\hspace{0pt}}m{#1}}
\newcolumntype{Y}{>{\centering\arraybackslash}X}
\newcolumntype{Z}{>{\raggedleft\arraybackslash}X}
\newcommand{\fmm}{\displaystyle} 
\newcommand{\cn}[1]{\underline{#1}} 
\newcommand{\hlaplace}{\quad\laplace\quad}
\newcommand{\hLaplace}{\quad\Laplace\quad}
\newcommand{\infint}{\int_{-\infty}^{+\infty}}
\newcommand{\infiint}{\iint_{-\infty}^{+\infty}}
\newcommand{\limint}{\lim_{T\rightarrow \infty} \frac{1}{T} \int_{-T/2}^{T/2}}
\newcommand{\bedeq}{\mathrel{\stackrel{\makebox[0pt]{\mbox{\normalfont\tiny WSS}}}{=\joinrel=}}}

\renewenvironment{description}{\color{cGray}}{}
\newenvironment{definition}{\color{cGray}}{}
\newcommand{\cdef}[1]{\begin{definition}#1\end{definition}}


\newcommand{\vLaplace}[1][]{\mbox{\setlength{\unitlength}{0.1em}%
        \begin{picture}(10,20)%
          \put(3,2){\circle{4}}%
          \put(3,4){\line(0,1){12}}%
          \put(3,18){\circle*{4}}%
          \put(10,7){#1}
        \end{picture}%
       }%
 }%

\newcommand{\vlaplace}[1][]{\mbox{\setlength{\unitlength}{0.1em}%
        \begin{picture}(10,20)%
          \put(3,2){\circle*{4}}%
          \put(3,4){\line(0,1){12}}%
          \put(3,18){\circle{4}}%
          \put(10,7){#1}
        \end{picture}%
       }%
 }%         
 
           
 
 
 
\renewcommand{\arraystretch}{1.5}


\newenvironment{mtabular}[1] {
  \renewcommand{\arraystretch}{2}
  
  \begin{tabular}{#1}
}  
{
  \end{tabular}
  
  \renewcommand{\arraystretch}{1.5}
}

\newenvironment{dtabular} {
  \begin{tabular}{>{\begin{definition}}l<{\end{definition}} >{\begin{definition}}l<{\end{definition}}}
}  
{
  \end{tabular}
}

\newenvironment{ddtabular} {
  \begin{center}
  \begin{tabular}{>{\begin{definition}}l<{\end{definition}} >{\begin{definition}}l<{\end{definition}} | >{\begin{definition}}l<{\end{definition}} >{\begin{definition}}l<{\end{definition}}}
}  
{
  \end{tabular}
  \end{center}
}


%configure tikz
%system description
\usetikzlibrary{shapes,arrows}
\tikzstyle{block} = [draw, rectangle, minimum height=3em, minimum width=4em]
\tikzstyle{input} = [coordinate]
\tikzstyle{output} = [coordinate]
\tikzstyle{pinstyle} = [pin edge={to-,thin,black}]
\tikzstyle{sum} = [draw, circle, node distance=1em, minimum height=1.5em]
\tikzset{>=latex}
\tikzset{%
  block/.style    = {draw, thick, rectangle, minimum height = 3em,
    minimum width = 3em},
  sum/.style      = {draw, circle, node distance = 1.5cm}, % Adder
  input/.style    = {coordinate}, % Input
  output/.style   = {coordinate} % Output
}


%externalize TIKZ
%\usetikzlibrary{external}
%\tikzexternalize[prefix=tikz/]

%lstlisting

\lstset{
  backgroundcolor=\color{lbcolor},
  tabsize=2,    
% rulecolor=,
  language=[GNU]C++,
  basicstyle=\scriptsize,
  upquote=true,
  aboveskip={1.5\baselineskip},
  columns=fixed,
  showstringspaces=false,
  extendedchars=false,
  breaklines=true,
  prebreak = \raisebox{0ex}[0ex][0ex]{\ensuremath{\hookleftarrow}},
  frame=single,
  numbers=none,
  showtabs=false,
  showspaces=false,
  showstringspaces=false,
  identifierstyle=\ttfamily,
  keywordstyle=\color{cBlue}
  commentstyle=\color{cGreen},
  stringstyle=\color{cRed},
  numberstyle=\color{black},
% \lstdefinestyle{C++}{language=C++,style=numbers}’.
}
\lstset{
  backgroundcolor=\color{lbcolor},
  tabsize=2,
  language=C++,
  captionpos=b,
  tabsize=3,
  frame=lines,
  numbers=none,
  numberstyle=\tiny,
  numbersep=5pt,
  breaklines=true,
  showstringspaces=false,
  basicstyle=\ttfamily,
  identifierstyle=\color{cOrange},
  keywordstyle=\color{cBlue},
  commentstyle=\color{cGreen},
  stringstyle=\color{cRed}
}

\lstdefinelanguage{makefile}{
  morekeywords={cc,CFLAGS,LFLAGS,OBJ,EXE},
  morecomment=[l]{\#}
}

\lstdefinestyle{makefile}{
  language=makefile,
  basicstyle=\ttfamily,
  keywordstyle=\color{cBlue},
  commentstyle=\color{cGreen},
  frame=lines,
  numbers=none,
  backgroundcolor=\color{lbcolor}
}

%header & footer
\pagestyle{fancy}
\lhead{Tibor Schneider}
\rhead{Seite \thepage}
\cfoot{\today} 

\renewcommand{\headrulewidth}{0.4pt}
\renewcommand{\footrulewidth}{0.4pt}
%Title of Document
\chead{Nachrichtentechnik 2 - Formelsammlung} 

\begin{document}

\begin{twocolumn}
 
\section{Wahrscheinlichkeitsrechnung}

\subsection{Begriffe}
Bei \textbf{a-priori} sind die Wahrscheinlichkeiten bereits vor dem ersten Durchführen des
Experimentes bekannt. Es kann auch erzwungen werden. Beispiel: Fairer Würfel. Bei
\textbf{a-posteriori} werden die Wahrscheinlichkeiten ermittelt, indem das Experiment
durchgeführt wird.

Wenn 2 Ereignisse \textbf{Unabhängig} sind, so verändert das Auftreten eines Ereignis
nicht die Wahrscheinlichkeit des anderen Ereignis: $P(B \vert A) = P(B)$. 

Ein \textbf{Elementarereignis} ist ein Ereignisraum mit nur einem Ereignis.

\subsection{Definition Zufallsexperiment}
\begin{description}
  \begin{tabularx}{\columnwidth}{XX}
    \begin{tabular}{ll}
      $S$ & Ereignisraum ($\Omega$) \\
      $n$ & Anzahl Ereignisse in $S$ \\
      $\lambda_k$ & Ereignis \\
      $A \in S$ & Teilmenge von $S$ \\
      $F_X(x)$ & Verteilungsfunktion \\
      $f_X(x)$ & W'keitsdichte-Funktion \\
      $\mu_X$ & Erwartungswert \\
      $\sigma_X^2$ & Varianz
    \end{tabular} &   
    \begin{tabular}{ll} 
      $\overline{A} = S \setminus A$ & Komplementärereignis \\
      $C = A \cup B$ & Ereignisse in A \textbf{und} B \\
      $C = A \cap B$ & Ereignisse in A \textbf{oder} B\\
      $C = S$ & Sichereres Ereignis \\
      $C = \emptyset$ & Unmögliches Ereignis \\
      $B \subset A$ & B ist Teil von A\\
      $A \cap B = \emptyset$ & A und B sind \textbf{disjunkte} \\
    \end{tabular}
  \end{tabularx}  
\end{description}

\subsection{Wahrscheinlichkeiten}
\begin{tabular}{c} 
  $\fmm 0 \leq P(A) \leq 1 \qquad \fmm P(S) = 1 \qquad P(\emptyset) = 0$ \\
  $\fmm P(\overline{A}) = 1 - P(A)$ \\
  $\fmm P(B) \leq P(A) \quad \text{für } B \subset A$ \\
  $\fmm P(A \cup B) = P(A) + P(B) \quad \text{für disjunkte A und B}$ \\
  $\fmm P(A \cup B) = P(A) + P(B) - P(A \cap B)$ \\
  $\fmm P(B \vert A) = \frac{P(A \cap B)}{P(A)} = P(A \vert B) \cdot \frac{P(B)}{P(A)}$ \\
  $\fmm P(A \cap B) = P(A) \cdot P(B) \quad \text{wenn A und B Unabhängig}$ \\
\end{tabular}

\subsection{Verteilungsfunktion}
\begin{tabular}{c}
  $\fmm F_X(x) = P(X \leq x) \quad \text{für } x \in (-\infty, \infty), \quad
  F_X(-\infty) = 0 \quad F_X(\infty) = 1$ \\
  $\fmm P(a < x \leq b) = F_X(b) - F_X(a) \qquad P(X > a) = 1 - F_X(a)$ \\
  $\fmm f_X(x) = \frac{dF_X(x)}{dx} \geq 0 \quad F_X(x) = \int_{-\infty}^x f_X(\tilde{x})
  \cdot d\tilde{x} \qquad \int_{-\infty}^{\infty}f_X(x) \cdot dx = 1$
\end{tabular}

\subsection{Mehrdimensionale Zufallsvariablen}
\begin{definition}
  \begin{tabularx}{\columnwidth}{XX}
    \begin{tabular}{ll}
      $F_{XY}(x,y)$ & W'Keit von 2 ZV
      \\
      $p_{XY}(x,y)$ &
      Verbundsw'keitsf.\\
      $Z = g(X,Y)$ &
      Kombinierte ZV \\
    \end{tabular}&
    \begin{tabular}{l}
      $F_{XY}(x,y) = P(X \leq x, Y \leq y)$\\
      $p_{XY}(x,y) = P(X=x_i, Y=y_k)$  \\
    \end{tabular}
  \end{tabularx}
\end{definition} 

\begin{tabularx}{\columnwidth}{ZX}
  (1) Wenn statistisch Unabhängig: & $\begin{array}{l} F_{XY}(x,y) = F_X(x) \cdot F_Y(y)
  \\
  p_{XY}(x,y) = p_X(x) \cdot p_Y(y)  \end{array}$ \\
  Randverteilungsfunktion: & $\begin{array}{l} F_X(x) = F_{XY}(x, +\infty) \\ F_X(x) =
  F_{XY}(x, +\infty) \end{array}$ \\
  Verbundsw'keitsfunktion & $\begin{array}{l} \fmm p_X(x) = \sum_{y_k} p_{XY}(x,y_k) \\
  \fmm p_Y(y) = \sum_{x_i} p_{XY}(x_i,y) \end{array}$ \\
\end{tabularx}

\subsubsection{Verbundw'keitsdichtefunktion}
\begin{tabularx}{\columnwidth}{YY}
  \multicolumn{2}{c}{
  $\fmm f_{XY}(x,y) = \frac{\partial^2 F_{XY}(x,y)}{\partial x \cdot \partial y} =
  f_X(x) \cdot f_Y(y) \quad \text{falls X, Y statistisch Unabhängig}$}
  \\
  $\fmm f_X(x) = \int_{-\infty}^{\infty}f_{XY}(x,y) dy$ &
  $\fmm f_Y(y) = \int_{-\infty}^{\infty}f_{XY}(x,y) dx$ \\
\end{tabularx}

\subsubsection{Erwartungswert (n-tes Moment) \& Varianz}
\begin{tabular}{ll}
  \multicolumn{2}{l}{$\fmm \mu_X = \E[X] = \sum_{i} x_i \cdot p_X(x_i) =
  \int_{-\infty}^\infty x \cdot f_X(x) \cdot dx = ``\text{Wert } \cdot \text{W'keit}"$}\\
  Linearität des Erwartungswert & 
  $\E[X+Y] = \E[X] + \E[Y] $ \\
  & $\E[cX] = c\E[X]$ \\
\end{tabular}

\begin{tabularx}{\columnwidth}{Y}
\textbf{Erwartungswert \& Moment} \\
$\fmm \E[Y] = \E[g(X)] = \sum_{i} g(x_i) \cdot p_X(x_i) = \int_{-\infty}^\infty g(x) \cdot
f_X(x) \cdot dx$ \\
$ \E[Z] = \E[g(X,Y)] = \sum_i \left( \sum_j g(x_i,y_j) \cdot p_{XY}(x_i,y_i) \right)$  \\
$ \fmm \E[Z] = \E[g(X,Y)] = 
\int_{-\infty}^\infty \left( \int_{-\infty}^\infty g(x,y) \cdot f_{XY}(x,y) dx \right) dy$
\\ 
$ \fmm \text{n-tes Moment: } E(X^n) = \sum_i x_i^n \cdot p_X(x_i) =
\int_{-\infty}^{\infty} x^n \cdot f_X(x) \cdot dx$ \\
\textbf{Varianz} \\
$\fmm \sigma_X^2 = \Var(X) = E \left[ X^2 \right] - \mu_X^2 = \sum_i(x_i - \mu_X)^2 \cdot p_X(x_i)$ \\
$\fmm \sigma_X^2 = \Var(X) = \int_{-\infty}^{\infty} (x-\mu_X)^2 \cdot f_X(x) \cdot dx$ \\
\end{tabularx}

\subsubsection{(k,n)-tes Moment $m_{kn}$, Korrelation $m_{11}$}
\begin{tabularx}{\columnwidth}{Y}
  $\fmm m_{kn} = E \left[ X^k Y^n \right] = \sum_j \left( \sum_i x_i^k y_j^n \cdot
  p_{XY}(x_i,y_i)\right)$ \\
  $\fmm m_{kn} = E\left[X^k Y^n\right] = \int_{-\infty}^\infty \left(
  \int_{-\infty}^\infty x^k y^n \cdot f_{XY}(x,y) \cdot dx \right) dy$ \\
  $\fmm \textbf{Spezialfall: Korrelation} \quad m_{11} = \E[XY] $
\end{tabularx}

\subsubsection{Kovarianz $\sigma_{XY}$}
\begin{tabularx}{\columnwidth}{Y}
  $\fmm \sigma_{XY} = \Cov(X,Y) = E\left[ (X-\mu_X) \cdot (Y-\mu_Y) \right] $ \\
  $\fmm \sigma_{XY} = \E[XY] - \E[X] \cdot \E[Y] = m_{11} - \mu_X \cdot \mu_Y$ \\
  Die zwei Zufallsvariablen $X$ und $Y$ sind \textbf{unkorreliert}, falls $\sigma_{XY} =
  0$. Statistisch unabhängige Zufallsvariablen sind unkorreliert.
\end{tabularx}

\subsubsection{Korrelations-Koeffizient $\rho(PX)$}
$$\rho(X,Y) = \rho_{XY} = \frac{\sigma_{XY}}{\sigma_{X} \cdot \sigma_Y} \qquad
\abs{\rho_{XY}} \leq 1$$

\subsection{Verteilungen}
$$\binom{n}{k} = \frac{n!}{(n-k)! \cdot k!} \text{ (TR: ncr)}$$
\subsubsection{Binomialverteilung}
Ein Experiment mit 2 möglichen Ausgängen, welche mit der W'keit $p$, bzw. $1-p$
zutreffen, wird $n$ mal wiederholt. Die ZV $x$ ist die Anzahl, wie oft ein Ausgang
eintritt. 

\vspace{1em}

\begin{tabular}{@{}p{0.35\columnwidth}l@{}}
  \begin{tabular}{c}
    $\fmm \mu_X = n \cdot p \quad \sigma_x^2 = n p \cdot (1-p) $ \\
    $\fmm p_X(k) = \binom{n}{k} p^k \cdot (1-p)^{n-k}$ \\
    $\fmm F_X(x) = \sum_{k=0}^m \binom{n}{k} p^k \cdot (1-p)^{n-k}$ \\
    \textbf{Grafik:} $p=0.4$, $n=100$
  \end{tabular} & 
  \begin{tabular}{l}
    \begin{tikzpicture}
      \begin{axis}[xlabel=$x$, ylabel=$p_X(x)$, axis lines=middle, width=0.6\columnwidth,
                   height=0.35\columnwidth, ytick = {0}, axis line style = {-latex}]
        %\pgfplotsset{ticks=none}
        \addplot [draw=cRed, name path=f, thick] coordinates{
              (1,0.0000)(2,0.0000)(3,0.0000)(4,0.0000)(5,0.0000)(6,0.0000)(7,0.0000)(8,0.0000)(9,0.0000)(10,0.0000)(11,0.0000)(12,0.0000)(13,0.0000)(14,0.0000)(15,0.0000)(16,0.0000)(17,0.0000)(18,0.0000)(19,0.0000)(20,0.0000)(21,0.0000)(22,0.0001)(23,0.0001)(24,0.0003)(25,0.0006)(26,0.0012)(27,0.0022)(28,0.0038)(29,0.0063)(30,0.0100)(31,0.0151)(32,0.0217)(33,0.0297)(34,0.0391)(35,0.0491)(36,0.0591)(37,0.0682)(38,0.0754)(39,0.0799)(40,0.0812)(41,0.0792)(42,0.0742)(43,0.0667)(44,0.0576)(45,0.0478)(46,0.0381)(47,0.0292)(48,0.0215)(49,0.0152)(50,0.0103)(51,0.0068)(52,0.0042)(53,0.0026)(54,0.0015)(55,0.0008)(56,0.0004)(57,0.0002)(58,0.0001)(59,0.0001)(60,0.0000)(61,0.0000)(62,0.0000)(63,0.0000)(64,0.0000)(65,0.0000)(66,0.0000)(67,0.0000)(68,0.0000)(69,0.0000)(70,0.0000)(71,0.0000)(72,0.0000)(73,0.0000)(74,0.0000)(75,0.0000)(76,0.0000)(77,0.0000)(78,0.0000)(79,0.0000)(80,0.0000)(81,0.0000)(82,0.0000)(83,0.0000)(84,0.0000)(85,0.0000)(86,0.0000)(87,0.0000)(88,0.0000)(89,0.0000)(90,0.0000)(91,0.0000)(92,0.0000)(93,0.0000)(94,0.0000)(95,0.0000)(96,0.0000)(97,0.0000)(98,0.0000)(99,0.0000)(100,0.0000)
            }; 
        \path[name path=axis, draw=none] (axis cs:0,0) -- (axis cs:100,0);
        \addplot [thick, color=clRed, fill=cRed, fill opacity=0.2] fill between[of=f and
        axis, soft clip={domain=-1:1}];
      \end{axis}
    \end{tikzpicture}
  \end{tabular}
\end{tabular}

\subsubsection{Poissonverteilung}
Eintreten eines seltenen Ereignis ($p \ll 1$) innerhalb einer gewissen Zeitspanne. In NaT
wird es jedoch meisstens in einer gewissen Länge $n$ betrachtet.

\vspace{1em}
\begin{tabular}{@{}p{0.35\columnwidth}l@{}}
  \begin{tabular}{c}
    $\mu_X = np = \lambda \qquad \sigma_X^2 = np = \lambda$ \\
    $\fmm p_X(k) = e^{-\lambda} \cdot \frac{\lambda^k}{k!}$\\
    $\fmm F_X(x) = e^{-\lambda} \cdot \sum_{k=0}^x \frac{\lambda^k}{k!}$ \\
    \textbf{Grafik: }$n=1000$, $p=0.005$  
  \end{tabular} & 
  \begin{tabular}{l}
    \begin{tikzpicture}
      \begin{axis}[xlabel=$x$, ylabel=$F_X(x)$, axis lines=middle, width=0.6\columnwidth,
                   height=0.35\columnwidth, ymin = 0, ymax = 1.1, ytick = {0, 1}, axis line
                   style = {-latex}]
        %\pgfplotsset{ticks=none}
        \addplot [const plot,color=cRed, thick]
        coordinates{ (0,0.04043)(1,0.12465)(2,0.26503)(3,0.44049)(4,0.61596)(5,0.76218)(6,0.86663)(7,0.93191)
              (8,0.96817)(9,0.98630)(10,0.99455)(11,0.99798)(12,0.99930)(13,0.99977)(14,0.99993)(15,0.99998)
              (16,0.99999)(17,1.00000)(18,1.00000)(19,1.00000)(20,1.00000)
        };

      \end{axis}
    \end{tikzpicture}
  \end{tabular}
\end{tabular}

\newpage

\subsubsection{Normalverteilung (Gaussverteilung)}
Anwendbar bei einer ZV, welche aus vielen Faktoren besteht. Beispiel: thermische
Rauschquelle.

\begin{tabular}{@{}p{0.35\columnwidth}l@{}}
  \begin{tabular}{c}
    $\fmm f_X(x) = \frac{1}{\sigma \sqrt{2\pi}}\cdot e^{\frac{-(x-\mu)^2}{2\sigma^2}}$ \\
    $\fmm F_X(x) = \frac{1}{\sqrt{2\pi}} \int_{-\infty}^{\frac{x-\mu}{\sigma}}
    e^\frac{-\xi^2}{2} \cdot d\xi $ \\
    Normalisierung: $\fmm z := \frac{x-\mu}{\sigma}$
  \end{tabular} & 
  \begin{tabular}{l}
    \begin{tikzpicture}
      \begin{axis}[xlabel=$x$, axis lines=middle, width=0.65\columnwidth,
                   height=0.35\columnwidth, xmin=-3.2, xmax=3.2, ymin=-0.1,ymax=1.2,
                   ytick = {0,0.5,1}, xtick = {-1,0, 0.6,1, 1.5}, xticklabels =
                   {$-\sigma$, $\mu$, $x_1$, $\sigma$, $x_2$}, legend pos=north west, legend
                   style={draw=none}, axis line style = {-latex}]
        %\pgfplotsset{ticks=none}

        \addplot [name path=f,color=cRed, thick] coordinates{
          (-4.0,0.0001)(-3.9,0.0002)(-3.8,0.0003)(-3.7,0.0004)(-3.6,0.0006)(-3.5,0.0009)(-3.4,0.0012)(-3.3,0.0017)(-3.2,0.0024)(-3.1,0.0033)(-3.0,0.0044)(-2.9,0.0060)(-2.8,0.0079)(-2.7,0.0104)(-2.6,0.0136)(-2.5,0.0175)(-2.4,0.0224)(-2.3,0.0283)(-2.2,0.0355)(-2.1,0.0440)(-2.0,0.0540)(-1.9,0.0656)(-1.8,0.0790)(-1.7,0.0940)(-1.6,0.1109)(-1.5,0.1295)(-1.4,0.1497)(-1.3,0.1714)(-1.2,0.1942)(-1.1,0.2179)(-1.0,0.2420)(-0.9,0.2661)(-0.8,0.2897)(-0.7,0.3123)(-0.6,0.3332)(-0.5,0.3521)(-0.4,0.3683)(-0.3,0.3814)(-0.2,0.3910)(-0.1,0.3970)(0.0,0.3989)(0.1,0.3970)(0.2,0.3910)(0.3,0.3814)(0.4,0.3683)(0.5,0.3521)(0.6,0.3332)(0.7,0.3123)(0.8,0.2897)(0.9,0.2661)(1.0,0.2420)(1.1,0.2179)(1.2,0.1942)(1.3,0.1714)(1.4,0.1497)(1.5,0.1295)(1.6,0.1109)(1.7,0.0940)(1.8,0.0790)(1.9,0.0656)(2.0,0.0540)(2.1,0.0440)(2.2,0.0355)(2.3,0.0283)(2.4,0.0224)(2.5,0.0175)(2.6,0.0136)(2.7,0.0104)(2.8,0.0079)(2.9,0.0060)(3.0,0.0044)(3.1,0.0033)(3.2,0.0024)(3.3,0.0017)(3.4,0.0012)(3.5,0.0009)(3.6,0.0006)(3.7,0.0004)(3.8,0.0003)(3.9,0.0002)(4.0,0.0001)
          };
        \addlegendentry{$f_X(x)$};
        \addplot [color=cGreen, thick] coordinates{
          (-4.0,0.0000)(-3.9,0.0000)(-3.8,0.0001)(-3.7,0.0001)(-3.6,0.0002)(-3.5,0.0002)(-3.4,0.0003)(-3.3,0.0005)(-3.2,0.0007)(-3.1,0.0010)(-3.0,0.0013)(-2.9,0.0019)(-2.8,0.0026)(-2.7,0.0035)(-2.6,0.0047)(-2.5,0.0062)(-2.4,0.0082)(-2.3,0.0107)(-2.2,0.0139)(-2.1,0.0179)(-2.0,0.0228)(-1.9,0.0287)(-1.8,0.0359)(-1.7,0.0446)(-1.6,0.0548)(-1.5,0.0668)(-1.4,0.0808)(-1.3,0.0968)(-1.2,0.1151)(-1.1,0.1357)(-1.0,0.1587)(-0.9,0.1841)(-0.8,0.2119)(-0.7,0.2420)(-0.6,0.2743)(-0.5,0.3085)(-0.4,0.3446)(-0.3,0.3821)(-0.2,0.4207)(-0.1,0.4602)(0.0,0.5000)(0.1,0.5398)(0.2,0.5793)(0.3,0.6179)(0.4,0.6554)(0.5,0.6915)(0.6,0.7257)(0.7,0.7580)(0.8,0.7881)(0.9,0.8159)(1.0,0.8413)(1.1,0.8643)(1.2,0.8849)(1.3,0.9032)(1.4,0.9192)(1.5,0.9332)(1.6,0.9452)(1.7,0.9554)(1.8,0.9641)(1.9,0.9713)(2.0,0.9772)(2.1,0.9821)(2.2,0.9861)(2.3,0.9893)(2.4,0.9918)(2.5,0.9938)(2.6,0.9953)(2.7,0.9965)(2.8,0.9974)(2.9,0.9981)(3.0,0.9987)(3.1,0.9990)(3.2,0.9993)(3.3,0.9995)(3.4,0.9997)(3.5,0.9998)(3.6,0.9998)(3.7,0.9999)(3.8,0.9999)(3.9,1.0000)(4.0,1.0000)
          };


        \addlegendentry{$F_X(x)$};

        \draw [dashed] (axis cs:-1,0) -- (axis cs:-1,0.2420);
        \draw [dashed] (axis cs:1,0.2420) -- (axis cs:1,0);

        \path[name path=axis] (axis cs:0,0) -- (axis cs:0.6,0);
        \addplot [thick, color=clRed, fill=cRed, fill opacity=0.2] fill between[of=f and
        axis, soft clip={domain=0:0.6}];


        \path[name path=axis] (axis cs:1.5,0) -- (axis cs:4,0);
        \addplot [thick, color=clRed, fill=cBlue, fill opacity=0.3] fill between[of=f and
        axis, soft clip={domain=1.5:4}];

        \draw [<-, color=cRed, >=latex] (axis cs:0.4,0.145) -- (axis cs:1.7, 0.5);
        \node [color=cRed, above right] at (axis cs:1.65,0.37) {$\erf(x)$};

        \draw [<-, color=cBlue, >=latex] (axis cs:1.7,0.045) -- (axis cs:1.9, 0.25);
        \node [color=cBlue, above right] at (axis cs:1.85,0.12) {$\Q(x)$};

      \end{axis}

    \end{tikzpicture}
  \end{tabular}
\end{tabular}

\begin{tabularx}{\columnwidth}{Y|Y}
   $\color{cBlue}\Q(x)$ \textbf{als Skalierung} & $\color{cRed}\erf(x)$ \textbf{als
   Matlab-Funktion:} \\ 
   $\fmm \Q(z) = \frac{1}{\sqrt{2\pi}} \int_z^\infty e^{-\frac{\xi^2}{2}} d\xi$&
  $\fmm \erf(x) = \frac{2}{\sqrt{\pi}} \int_0^x e^{-\xi^2} d\xi$ \\
  $\fmm F_X(x) = 1-\Q\left(\frac{x-\mu}{\sigma}\right) = 1-\Q(z)$&
  $\fmm F_X(x) = \frac{1}{2} \left( 1 + \erf \left( \frac{x - \mu}{\sqrt{2}\sigma}\right)
  \right)$ \\
  $\fmm \Q(-z) = -\Q(z)$
\end{tabularx}

\section{Zufallsprozesse}
\begin{definition}
\begin{tabularx}{\columnwidth}{XX}
    \begin{tabular}{ll}
      $X(t, \lambda)$ & Zufallsprozess \\ 
      \multicolumn{2}{l}{$X(t,\lambda_i) = x_i(t) \quad$ Musterfunktion} \\
      $X_k(\lambda)$ & ZV zum Zeitpunkt $t_k$ \\
      $F_{X_k}(x_k;t_k)$ & Verteilungsfunktion \\
      $f_{X_k}(x_k;t_k)$ & W'keitsdichtefunktion \\
      $\mu_X(t)$ & Erwartungswert von $X(t)$\\
      $R_{XX}(t_1,t_2)$ & Autokorrelation \\
      $C_{XX}(t_1,t_2)$ & Autokovarianz \\
      \multicolumn{2}{l}{$C_{XX}(t_1,t_2) = 0\quad$ ZV sind unkorreliert}  \\
    \end{tabular}&
    \begin{tabular}{l}
      $\fmm F_{X_k}(x_k;t_k) = P \{ X_k(t_k) \leq x_k \}$ \\
      $\fmm f_{X_k}(x_k;t_k) = \frac{\partial F_{X_k}(x_k;t_k)}{\partial x_k}$ \\
      $\fmm F_X(x_1, x_2;t_1, t_2) $ \\
      $ \qquad = P \{X(t_1) \leq x_1, X(t_2) \leq x_2 \}$ \\
      $\fmm f_X(x_1,x_2;t_1,t_2) = \frac{\partial^2 F_X(x_1,x_2;t_1,t_2)}{\partial x_1
      \partial x_2}$\\ 
      $\tau = t_1 - t_2$
    \end{tabular}
\end{tabularx}
\end{definition} 

\subsection{Statische Kennwerte}

\begin{tabular}{c}
  $\fmm \mu_x(t) = E \left[X(t) \right] = \infint x \cdot f_X(x;t) dx$ \\
  $\fmm R_{XX}(t_1,t_2) = E \left[ X(t_1)X(t_2) \right] = \infiint x_1 x_2 \cdot
  f_X(x_1,x_2;t_1,t_2) \cdot dx_1 \cdot dx_2$ \\
  $\fmm C_{XX}(t_1,t_2) = R_{XX}(t_1, t_2) - \mu_x(t_1) \cdot \mu_x(t_2)$
\end{tabular} 


\subsection{Stationarität}
Ein Zufallsprozess ist Stationär, wenn ihre statistischen Eigenschaften zeitlich
unverändert bleiben. Man unterschiedet streng und schwach stationäre Prozesse.

\subsubsection{Statische Eigenschaften}
\begin{tabularx}{\columnwidth}{Y}
  $\fmm \overline{x} = \left< x(t) \right> = \limint x(t) dt$ \\
  $\fmm \overline{R_{XX}}(\tau) = \left<x(t) \cdot x(t + \tau) \right> = \limint x(t)
  \cdot x(t + \tau) dt$ \\
  $\fmm \E[\overline{x}] = \limint \E \left[x(t) \right] dt = \limint \mu_x(t) dt$
\end{tabularx}

\subsubsection{Streng stationärer Zufallsprozess (SSS)}
\begin{tabularx}{\columnwidth}{XX}
  \begin{tabular}{p{0.45\columnwidth}}
    Alle statischen Eigenschaften eines SSS-Prozesses sind unabhängig von einer beliebigen
    Zeitverschiebung $t_c$: 
  \end{tabular}&
  \begin{tabular}{l}
    $E \left[X(t) \right] = \mu_x$ (unabhängig von $t$)\\
    $R_{XX}(t_1,t_2) = R_X(\tau)$ (unabhängig von $t$)\\
    $C_{XX}(t_1,t_2) = R_X(\tau) - \mu_x^2$
  \end{tabular}
\end{tabularx}

\subsubsection{Schwach stationärer Zufallsprozess (WSS)}
\begin{tabularx}{\columnwidth}{XX}
  \begin{tabular}{p{0.45\columnwidth}}
    Ein Zufallsprozess ist schwach stationär, sobald folgende Eigenschaften erfüllt sind: \\
  \end{tabular}&
  \begin{tabular}{l}
    $E \left[X(t) \right] = \mu_x$ (unabhängig von $t$) \\
    $R_{XX}(t_1,t_2) = R_X(\tau)$ (unabhängig von $t$)\\
    $C_{XX}(t_1,t_2) = R_X(\tau) - \mu_x^2 = C_{XX}(\tau)$ \\
    $f_X(x_1,x_2;t_1,t_2) = f_X(x_1,x_2;t_1+t_c,t_2+t_c)$ \\
  \end{tabular}
\end{tabularx}

\subsubsection{Ergodizität}
\begin{tabularx}{\columnwidth}{XX}
  \begin{tabular}{p{0.45\columnwidth}}
    Ein Stationärer Prozess ist ergodisch, wenn zusätzlich auch jede einzelne Musterfunktion
    dieselben Zeitmittelwerte besitzt, und wenn diese mit den statischen Kennwerten der Schar
    übereinstimmt. 
    \\
    \begin{tabular}{>{\color{cGray}}l>{\color{cGray}}l}
      $E \left[X(t)\right]$ & DC-Level \\
      $E \left[X(t)\right]^2$ & DC-Leistung \\
      $E \left[X^2(t)\right]$ & Gesamtleistung \\
      $\sigma_X^2(t)$ & AC-Leistung \\
      $\sugma_X(t)$ & RMS-Level \\
    \end{tabular}
  \end{tabular}&
  \begin{tabular}{l}
    $\fmm \overline{x_i} = \overline{x_i(t)} = E \left[x(t)\right] = \mu_X$ \\
    $\fmm E \left[\overline{x(t)}\right] = \mu_X$ \\
    $E\left[ X(t) \right] = \overline{x} = \left<x(t)\right>$ \\
    $E\left[ X(t) \right]^2 = \left< x(t) \right>^2 \quad $ \\
    $E\left[ X^2(t) \right] = R_{XX}(t) = \left< x^2(t) \right>$\\
    $\sigma_X^2(t) = \left<x^2(t) \right> - \left< x(t) \right>^2$ \\
    $\sigma_X(t) = \overline{\sigma_X}$ \\
  \end{tabular} 
\end{tabularx}

\subsubsection{Eigenschaften von stationären Zufallsprozessen}

Auto- und Kreuz\textbf{korrelation}sfunktion bei WSS Prozessen:
\vspace{0.5em}

\begin{tabularx}{\columnwidth}{ZX}
   & $R_{XX}(-\tau) = R_{XX}(\tau)$ \\
   $R_{XX}(\tau) = E\left[ X(t) X(t + \tau) \right]$ &
   $\abs{R_{XX}(\tau)} \leq R_{XX}(0)$ \\
   & $R_{XX}(0) = E\left[ X^2(x) \right]$ \\
\end{tabularx}

\vspace{0.5em}

\begin{tabularx}{\columnwidth}{ZX}   
   & $R_{XY}(-\tau) = R_XY(\tau)$ \\
   $R_{XY}(\tau) = E\left[ X(t) Y(t + \tau) \right]$ &
   $\abs{R_{XY}{\tau}} \leq \frac{1}{2} \left( R_{XX}(0) + R_{YY}(0) \right)$ \\
   & $R_{XY}(\tau) \leq \sqrt{R_{XX}(0) R_{YY}(0)}$ \\
\end{tabularx}

\vspace{1em}
Auto- und Kreuz\textbf{kovarianz} be WSS Prozessen:
\vspace{0.5em}

\begin{tabularx}{\columnwidth}{Y}
  $C_{XX}(\tau) = R_{XX}(\tau) - \mu_X^2$ \\
  $C_{XY}(\tau) = R_{XY}(\tau) - \mu_X \cdot \mu_Y$
  $C_{XY}(\tau) = 0 \longrightarrow \text{Zufallsprozesse sind zueinander unkorreliert}$
\end{tabularx}

\newpage
Spektrale \textbf{Leistung} bei WSS Prozessen:
\vspace{0.5em}

\begin{tabularx}{\columnwidth}{Y}
  $\fmm S_{XX}(\omega) = E \left[ \lim_{T\rightarrow \infty} \frac{1}{T} \cdot
  \abd{X(\omega)}^2 \right] = \lim_{T\rightarrow \infty} \frac{1}{T} \cdot E\left[
  \abs{X(\omega)}^2 \right]$ \\
  $\fmm S_{XX}(\omega) = \infint R_{XX}(\tau) \cdot e^{-j \omega \tau}d \tau \qquad
  R_{XX}(\tau) = \frac{1}{2\pi} \infint S_{XX}(\omega) \cdot e^{j \omega \tau} d\omega$
\end{tabularx}

\subsection{Übertragung von Zufallsprozessen über LTI-Systeme}
\begin{tabular}{cl}
  \begin{tabular}{c}
    $\fmm Y(t) = X(t) \ast h(t) = \infint h(u)X(t-u) du$ \\
    $\fmm \mu_Y(t) = h(t) \ast \mu_x(t) \bedeq H(0) \cdot \mu_X$ \\
  \end{tabular} &
  \begin{tabular}{l}
    \begin{tikzpicture}[auto, node distance=2cm,>=latex']
      \node [input, name=input] {};
      \node [block, right of=input] (system) {$h(t)$};
      \node [output, right of=system] (output) {};

      \draw [->] (input) -- node[name=x] {$X(t)$} (system);
      \draw [->] (system) -- node [name=y] {$Y(t)$} (output);
    \end{tikzpicture}
  \end{tabular}
\end{tabular}

\begin{tabular}{c}
  $\fmm R_{YY}(t,t + \tau) = \infiint h(u_1) \cdot h(u_2) \cdot R_{XX}(t-u_1,t + \tau-u_2 )
  \cdot du_1 du_2 $ \\
  $\fmm R_{YY}(\tau) \bedeq \infiint h(u_1) \cdot h(u_2) \cdot R_{XX}(\tau + u_1 - u_2)
  \cdot du_1 \cdot du_2$ \\
  Wenn $X(t)$ ein WSS-Prozess ist, ist $Y(t)$ auch ein WSS-Prozess.
\end{tabular}

\subsubsection{Leistungsdichtespektrum}
  \begin{tabular}{c}
  $\fmm S_{YY}(\omega) = \infint R_{YY}(\tau) \cdot e^{-j \omega \tau} d\tau =
  H^\ast(\omega) \cdot H(\omega) \cdot S_{XX}(\omega) = \abs{H(\omega)}^2 \cdot
  S_{XX}(\omega)$ \\
  $\fmm R_{YY}(\tau) = \frac{1}{2\pi} \infint \abs{H(\omega)}^2 \cdot S_{XX}(\omega) \cdot
  e^{j \omega \tau} d\omega$ \\
  $\fmm E\left[Y^2(t)\right] = R_{YY}(0) = \frac{1}{2\pi} \infint \abs{H(\omega)}^2 \cdot
  S_{XX}(\omega) d\omega$
\end{tabular}

\subsection{Spezielle Zufallsprozesse}
\subsubsection{Gauss'scher Zufallsprozess}
Zu jedem Zeitpunkt $t_i$ ist die Zufallsvariable $X(t_i)$ normalverteilt. Ist ein
gauss'scher Zufallsprozess WSS, ist er auch SSS. Ein gauss'scher Zufallsprozess $X(t)$ ist
am Ausgang eines LTi Systems $Y(t)$ auch gaussisch.

\begin{tabular}{c}
$\fmm f_x(x_1,x_2;t_1,t_2) = \frac{1}{2\pi \sigma_{x_1}
\sigma_{x_2}} \cdot e^{-\frac{(x_1-\mu_{x_1})^2}{2\sigma_{x_1}^2}} \cdot
e^{-\frac{(x_2-\mu_{x_2})^2}{2\sigma_{x_2}^2}}$ \\
$\fmm \text{Wenn } C_{XX}(t_1,t_2) = 0 \text{, dann ist } f_{XX}(x_1,x_2;t_1,t_2) =
f_X(x_1;t_1) \cdot f_X{x_2;t_2}$
\end{tabular}

\subsection{Spektren von Zufallssesquenzen berechnen}
\begin{enumerate}
  \item Stationärer Zufallsprozess wird vorausgesetzt: Musterfunktion darf um eine beliebige 
  Verschiebung $\Delta t$ verschoben werden. Zu jedem Zeitpunkt ist jede Sequenz denkbar.
  \item Eindimensionale (diskrete) Amplitudenverteilung: $p_X(x)$
  
  \begin{tikzpicture}
    \begin{axis}[width=5cm, height=3cm, xlabel=$x$, ylabel=$p$, axis lines=middle,
                 xmin=-1.5, xmax = 1.5, ymin = -0.5, ymax = 1.5, xtick = {-1,1},
                 xticklabels = {$-A$, $A$}, ytick = {1}, yticklabels = {$0.5$}]
      \draw [-*] (axis cs:-1,0) -- (axis cs:-1,1);
      \draw [-*] (axis cs: 1,0) -- (axis cs: 1,1);
    \end{axis}
  \end{tikzpicture} 
  \item Zweidimensionale (diskrete) Amplitudenverteilung: $p_{XX}(x_1, x_2)$. 
  {\color{cRed}Rot}: $\tau = 0$, {\color{cBlue}Blau}: $\tau > T_p$, dazwischen linearer Anstieg. 
  Danach muss die AKF berechnet werden: $\fmm R_{XX}(\tau) = \sum_i \sum_j x_i \cdot x_j \cdot p_{XX}(x_i, x_j)$
  
  \begin{tikzpicture}[x=1cm,y=1cm,z=0.5cm,>=stealth]
    % The axes
    \draw[->] (xyz cs:x=-1.5) -- (xyz cs:x=1.5) node[above] {$x_1$};
    \draw[->] (xyz cs:y=-0.5) -- (xyz cs:y=1.5) node[right] {$p$};
    \draw[->] (xyz cs:z=-1.5) -- (xyz cs:z=1.5) node[above] {$x_2$};

    \draw[dashed] (xyz cs:x=-1,z=1) -- (xyz cs:x=-1,z=-1) -- (xyz cs:x=1,z=-1) -- (xyz cs:x=1,z=1) -- (xyz cs:x=-1,z=1);
    \draw[draw = cRed, -*] (xyz cs:x=-1,z=-1) -- (xyz cs:x=-1,y=1,z=-1) node[above]{$0.5$};
    \draw[draw = cRed, -*] (xyz cs:x= 1,z= 1) -- (xyz cs:x= 1,y=1,z= 1) node[above]{$0.5$};

  \end{tikzpicture}
  \begin{tikzpicture}[x=1cm,y=1cm,z=0.5cm,>=stealth]
    % The axes
    \draw[->] (xyz cs:x=-1.5) -- (xyz cs:x=1.5) node[above] {$x_1$};
    \draw[->] (xyz cs:y=-0.5) -- (xyz cs:y=1.5) node[right] {$p$};
    \draw[->] (xyz cs:z=-1.5) -- (xyz cs:z=1.5) node[above] {$x_2$};

    \draw[dashed] (xyz cs:x=-1,z=1) -- (xyz cs:x=-1,z=-1) -- (xyz cs:x=1,z=-1) -- (xyz cs:x=1,z=1) -- (xyz cs:x=-1,z=1);
    \draw[draw = cBlue, -*] (xyz cs:x=-1,z=-1) -- (xyz cs:x=-1,y=0.5,z=-1) node[above]{$0.25$};
    \draw[draw = cBlue, -*] (xyz cs:x= 1,z= 1) -- (xyz cs:x= 1,y=0.5,z= 1) node[above]{$0.25$};
    \draw[draw = cBlue, -*] (xyz cs:x=-1,z= 1) -- (xyz cs:x=-1,y=0.5,z= 1) node[above]{$0.25$};
    \draw[draw = cBlue, -*] (xyz cs:x= 1,z=-1) -- (xyz cs:x= 1,y=0.5,z=-1) node[above]{$0.25$};

  \end{tikzpicture}
  \begin{tikzpicture}
    \begin{axis}[width=5cm, height=3cm, xlabel=$\tau$, ylabel=$R_{xx}$, axis lines=middle,
                 xmin=-1.5, xmax = 1.5, ymin = -0.5, ymax = 1.5, xtick = {-1,1},
                 xticklabels = {$-T_p$, $T_p$}, ytick = {1}, yticklabels = {$A^2$}]
      \draw [draw = cRed, thick] (axis cs:-1.5,0) -- (axis cs:-1,0) -- (axis cs:0,1) -- (axis cs:1,0) -- (axis cs:1.5,0);
    \end{axis}
  \end{tikzpicture}
  \item $\mathcal{F}(R_{XX}(\tau)) \Rightarrow S_{XX}(\omega)$ 
  
\end{enumerate}

\section{Rauschen in analogen Kommunikationssystemen:}

\begin{tikzpicture}[auto, node distance=3cm,>=latex', scale=0.7, every node/.style={transform shape}]
  \node [input, name=input] {};
  \node [block, right of=input] (transmitter) {transmitter};
  \node [block, right of=transmitter] (channel) {Channel};
  \node [sum, right of=channel, node distance = 3cm] (summation) {$\sum$};
  \node [input, above of=summation, node distance = 1cm, name=noise] {};
  \node [block, right of=summation] (filter) {Band Filter};
  \node [block, right of=filter] (detector) {Detector};
  \node [output, right of=detector] (output) {};

  \draw [->] (input) -- node[name=x] {$X(t)$} (transmitter);
  \draw [->] (transmitter) -- node[name=xc]{$X_c(t)$} (channel);
  \draw [->] (channel) -- node[name=yc]{$Y_c(t)$} (summation);
  \draw [->] (noise) -- node[name=n]{$N(t)$} (summation);
  \draw [->] (summation) -- (filter);
  \draw [->] (filter) -- node[name=yi]{$Y_i(t)$} (detector);
  \draw [->] (detector) -- node[name=yo]{$Y_o(t)$} (output);
\end{tikzpicture}

Das \begin{definition}Rauschsignal $N(t)$\end{definition} ist ein stationärer und ergodischer
Zufallsprozess, welcher mit $X(t)$ unkorreliert ist. Ausserdem gilt: $\E[N(t)] = 0, \quad \E[X(t) \cdot N(t)] = 0$. 
Der Frequenzgang $S_{nn}(\omega) = \frac{\eta}{2}$.

Das Nachrichtensignal ist ein \begin{definition}Zufallsprozess $X(t)$\end{definition} mit 
$\abs{X(t)} \leq 1$, bzw. $\abs{x_\lambda(t)} \leq 1$ für alle $\lambda$ des Ereignisraums 
$s$. Die \begin{definition}Leistung $S_X$\end{definition} $= S_X(t) = \E[X^2(t)] \leq 1$. 
Ausserdem ist das Signal $X(t)$ bandbeschränkt mit der 
\begin{definition}Bandbreite $B$\end{definition}.

\cdef{Addititive white gaussian noise: AWGN}

\begin{footnotesize}
\begin{tabular}{|c|c|c|c|}
  \hline
   & \textbf{Baseband} & \textbf{DSB-SC} & \textbf{AM} \\ \hline
  Eingangsnutzsignal $X_i(t)$ & 
  $X(t)$ & $X(t) \cdot A_c \cdot \cos(\omega_c t)$ & $A_c (1 + \mu X(t)) \cos(\omega_c t)$ \\\hline
  Leistung $S_i$ von $X_i(t)$ & $S_x$ & $\frac{1}{2} \cdot A_c^2 \cdot S_x$ & 
  $\frac{1}{2}A_c^2 (1 + \mu^2 S_x)$ \\ \hline
  Bandbreite von $X_i(t)$ & $B$ & $2B$ & $2B$ \\ \hline
  Leistung $S_{Ni}$ von $N_i(t)$ & $\eta B$ & $2 \eta B$ & $2 \eta B$ \\ \hline
  $\text{SNR}_i$ & $\fmm \frac{S_i}{\eta B}$ & $\fmm \frac{A_c^2 S_x}{4 \eta B}$ & 
  $\fmm \frac{A_c^2 (1 + \mu^2 S_x)}{4 \eta B}$ \\ \hline
  Ausgangsnutzsignal $X_o(t)$ & $X(t)$ & $A_c \cdot X(t)$ & $A_c \cdot \mu \cdot X(t)$ \\ \hline
  Leistung $S_o$ von $X_o(t)$ & $S_X$ & $A_c^2 S_x$ & $A_c^2 \cdot \mu^2 \cdot S_X$ \\ \hline
  Leistung $S_{No}$ von $N_o(t)$ & $\eta B$ & $2 \eta B$ & $2 \eta B$ \\ \hline
  $\text{SNR}_o$ & $\fmm \frac{S_i}{\eta B}$ & $\fmm \frac{A_c^2 S_X}{2 \eta B}$ & 
  $\fmm \frac{A_c^2 \mu^2 S_X}{2 \eta B}$ \\ \hline
  $\text{SNR}_o$ mit $\gamma$ & $\gamma$ & $\gamma$ & $\frac{\mu^2 S_X}{1 + \mu^2 S_X} 
  \cdot \gamma$ \\ \hline 
\end{tabular} 

\begin{tabular}{|c|c|c|}
  \hline & \textbf{PM} & \textbf{FM} \\ \hline
  Eingangsnutzsignal $X_i(t)$ & $A_c \cdot \cos(\omega_c t + k_p \cdot X(t)) \quad$ & 
  $A_c \cos(\omega_c t + k_f \int_{-\infty}^t X(\tau) d\tau)$ \\\hline
  Leistung $S_i$ von $X_i(t)$ & $\frac{1}{2} A_c^2$ & $\frac{1}{2} A_c^2$ \\ \hline
  Bandbreite von $X_i(t)$ & $2(D + 1)B$ & $2(D + 1) B$ \\ \hline
  Leistung $S_{Ni}$ von $N_i(t)$ & $2(D + 1)\eta B$ & $2(D + 1) \eta B$ \\ \hline
  Ausgangsnutzsignal $X_o(t)$ & $\fmm \frac{A_c^2}{4(D+1)\eta B}$ & 
  $\fmm \frac{A_c^2}{4(D+1)\eta B}$ \\ \hline
  Leistung $S_o$ von $X_o(t)$ & $k_p \cdot X(t)$ & $k_f <cdot X(t)$ \\ \hline
  $\text{SNR}_i$ & $k_p^2 S_X$ & $k_f^2 S_X$ \\ \hline
  Leistung $S_{No}$ von $N_o(t)$ & $\fmm \frac{2}{A_c^2} \eta B$ & 
  $\fmm \frac{2(2\pi B)^2}{3A_c^2} \eta B$ \\ \hline
  $\text{SNR}_o$ & $\fmm \frac{k_p^2 A_c^2 S_x}{2 \eta B}$ & 
  $\fmm \frac{3 D^2 A_c^2 S_X}{2 \eta B}$ \\ \hline
  $\text{SNR}_o$ mit $\gamma$ & $k_p^2 \cdot S_x \cdot \gamma $ & 
  $3 D^2 \cdot S_X \cdot \gamma$ \\ \hline
\end{tabular}
\end{footnotesize}


\subsection{Rauschen bei Winklemodulierten Systemen}

Bei den PM- / FM-Systemen gilt für das \cdef{Hubverhältnis $D$} $ = \frac{\Delta f}{B_m} = \frac{\delta \omega}{W_m}$ 

\begin{tabular}{c}
  $\fmm Y_i(t) = A_c \cos(\omega_c t + \varphi(t)) + n_i(t) \quad \text{mit} \quad S_i = \frac{1}{2} A_c^2$ \\
  $\fmm \varphi_{PM} = k_p \cdot X(t) \qquad \varphi_{FM}(t) = k_f \int_{-\infty}^t X(\tau) d\tau$\\
  $\fmm Y_{o_{PM}}(t) = \Theta(t) = k_p X(t) + \frac{n_s(t)}{A_c} \qquad Y_{o_{FM}}(t) = \frac{d\Theta(t)}{dt} = k_f X(t) + \frac{n_s(t)}{A_c}$
\end{tabular}


\subsection{Additive White Gaussian Noise AWGN}
$$S_{XX}(\omega) = \frac{\eta}{2}, \qquad R_{XX}(\tau) = \frac{\eta}{2} \cdot \delta(\tau), \qquad \eta = \frac{\sigma^2}{B}$$
$$\text{Leistung eines gefilterten AWGN: } \quad S_N = \frac{1}{2\pi} \int_{-\infty}^\infty \frac{\eta}{2} \left[ H_{LPF}(\omega) \right]^2 \cdot d\omega$$

\subsection{Farbige Rauschsignale}
\begin{tabular}{ll}
  {Pink noise}: $S_{XX}(\omega) = c \cdot \frac{1}{\omega}$ & 
  {Brown noise}: $S_{XX}(\omega) = c \cdot \frac{1}{\omega^2}$ \\
  {Blue noise}: $S_{XX}(\omega) = c \cdot \omega$ & 
  {Purple noise}: $S_{XX}(\omega) = c \cdot \omega^2$ 
\end{tabular}

\section{Optimaler Detektor}
Der Empfänger soll so optimiert werden, dass er den übertragenen degitalen Datenstrom 
die Bitfehlerrate $P_e$ minimiert. Also muss die Signal-Noise-Ratio klein sein. Das \cdef{Rauschsignal $n(t)$} 
ist ein Mittelwertfreies AWGN (additive white gaussian noise). der Abtaster erzeugt das diskrete Signal 
$z(nT) = a_i(nT) + n_0(nT)$ (zu en Zeitpunkten $nT$). 

\begin{center}\begin{tikzpicture}[auto, node distance=3cm,>=latex', scale=0.7, every node/.style={transform shape}]
  \node [input, name=input] {};
  \node [sum, right of=input, node distance = 2cm] (summation) {$\sum$};
  \node [input, above of=summation, node distance = 1cm, name=noise] {};
  \node [block, right of=summation] (filter) {Linear Filter};
  \node [block, right of=filter] (switch) {SW};
  \node [block, right of=switch] (comparator) {Komparator};
  \node [output, right of=comparator] (output) {};

  \draw [->] (input) -- node[name=s] {$s_i(t)$} (summation);
  \draw [->] (noise) -- node[name=n] {$n(t)$} (summation);
  \draw [->] (summation) -- node[name=r] {$r(t)$} (filter);
  \draw [->] (filter) -- node[name=z] {$z(t)$} (switch);
  \draw [->] (switch) -- node[name=znt]{$z(nT)$} (comparator);
  \draw [->] (comparator) -- node[name=yo]{$y(t)$} (output);
\end{tikzpicture} \end{center}

$$s_i(t) = \left\{ \begin{array}{lll} s_1(t) & 0 \leq t \leq T & \text{for binary 1} \\ 
s_2(t) & 0 \leq t \leq T & \text{for binary 0}\end{array} \right. \quad r(t) = s_i(t) + n(t)$$

\subsection{Detektor}
Der Detektor entscheidet durch eine Schätzung, welches Signal höchstwahrscheinlich gesendet wurde.
bei Soft-Descision wird anstatt fester Entscheidung eine W'keit für die beiden Symbolwerte bestimmt
und verarbeitet.

$$\begin{array}{ll}
  \text{Hypothese } H_1 \text{ (falls } z(nT) > \lambda \text{ ):} & s_1 \text{ wurde gesendet} \\
  \text{Hypothese } H_1 \text{ (falls } z(nT) < \lambda \text{ ):} & s_2 \text{ wurde gesendet} \\
\end{array}
$$

Ein \cdef{Bitfehler} ist, wenn das falsche Bit erkannt wird. Bei Soft-Descision bildet die 
W'keit ein Anhaltspunkt für mögliche Bitfehler. bei Hard-Descision kann mit folgender 
Fehlerw'keit gerechnet werden:
$$P_e = P(H_2 | s_1) <cdot P(s_1) + P(H_1 | s_2) \cdot P(s_2)$$
$$P_e = \frac{1}{2} \left[ P(H_2 | s_1) + p(H_1 | s_2) \right], \quad \text{wenn: } P(s_1) = P(s_2) = \frac{1}{2} $$

\subsection{Maximum Likelihood Detector}

\cdef{$z(nT)$} sei die Verteilung in Funktion von $s_1$ und $s_2$. 

$$\begin{array}{ll}
\text{Hypothese } H_1: & \Lambda(z) = \frac{f(z|s_1)}{f(z|s_2)} > \frac{P(s_2)}{P(s_1)} \\
\text{Hypothese } H_2: & \Lambda(z) = \frac{f(z|s_1)}{f(z|s_2)} < \frac{P(s_2)}{P(s_1)} \\
\end{array}$$

Beim vereinfachten ML Detector wird vorausgesetzt, dass $P(s_1) = P(s_2) = \frac{1}{2}$. Somit wird
$\frac{P(s_2)}{P(s_1)} = 1$.

\subsection{Beispiel: verrauschtes NRZ-Signal}
ein Signal $s$ mit den beiden möglichen Werten $s_1 = a_1, s_2 = a_2$ wird durch ein weisses Rauschen 
verfälscht:
$$f_{n_0}(\xi) = \frac{1}{\sqrt{2\pi} \sigma_{n_0}} \cdot e^{\frac{-\xi^2}{2 \sigma_{n_0}^2}} $$

$$\Longrightarrow f(z|s_1) = \frac{1}{\sqrt{2\pi} \sigma_{n_0}} \cdot e^{\frac{-(z-a_1)^2}{2\sigma_{n_0}^2}}, 
\qquad f(z|s_2) = \frac{1}{\sqrt{2\pi} \sigma_{n_0}} \cdot e^{\frac{-(z-a_2^2}{2\sigma_{n_0}^2}}$$

Somit ist die Wahrscheinlichkeit, dass die falsche Hypothese gewählt wird:

$$P(H_1|s_1) = \int_{-\infty}^{\lambda_0} f(z|s_1) dz, \qquad
P(H_1|s_1) = \int_{-\infty}^{\lambda_0} f(z|s_1) dz$$

Mit den Vereinfachungen, dass $P(s_1) = P(s_2) = 0.5$ und $\lambda_0 = \frac{a_1 + a_2}{2} $, dann ist 
die Wahrscheinlichkeit \cdef{$P_e$}, dass die falsche Hypothese gewählt wurde: 
$$P_e = Q\left(\frac{a_1-a_2}{2 \sigma_{n_0}}\right)$$

\subsection{Matched Filter}

Das \cdef{Lineare Filter $H(\omega)$} soll die Fehler-WSK zum \cdef{Zeitpunkt $T$} minimieren. 

\begin{tabular}{ll}
  \begin{tabular}{ll}
    \cdef{$H(\omega)$} & \cdef{Übertragungsfunktion des Filters} \\
    \cdef{$s(t)$} & \cdef{Differenzsignal $s_1 - s_2$}\\
    \cdef{$a_1$, $a_2$} & \cdef{Signalwerte von $s_1, s_2$} \\
    \cdef{$S(\omega)$} & \cdef{Spektrale Leistung des Signals} \\
    \cdef{$E_d$} & \cdef{Energie des Differenzsignals $s(t)$}\\
    \cdef{$P_e$} & \cdef{Fehlerwahrscheinlichkeit} \\
  \end{tabular} &
  
  \begin{tabular}{c}
    $\fmm H(\omega) = S^\ast(\omega) \cdot e^{-j \omega T}$ \\
    $\fmm \vLaplace$ \\
    $\fmm h(t) = \left\{ \begin{array}{ll} s(T-t) & 0 \leq t \leq T \\ 0 & \text{sonst} \end{array} \right.$ \\
    $\fmm \left(\frac{S}{N}\right)_{o_{max}} = \frac{(a_1 - a_2)^2}{\sigma_{n_0}^2} = \frac{2E_d}{\eta}$ \\
    $\fmm E_d = \int_0^T \left[ s_1(t) - s_2(t) \right]^2 dt$ \\
    $\fmm P_e = Q\left(\frac{a_1 - a_2}{2 \sigma_{n_0}}\right) = Q\left(\sqrt{\frac{E_d}{2\eta}}\right)$
  \end{tabular}
\end{tabular}

\subsubsection{Variante zum matched-Filter: Korrelator}
zum Sample Zeitpunkt $t = T$ ist die Antwort $z(t)$ eines Matched Filters und diejenigen eines \textbf{Korrelators} identisch. 
ein Korrelator multipliziert zwei Signale $s(t)$ und $r(t)$ und integriert das Resultat. 
$$\underbrace{z(t) = r(t) \ast h(t) = \int_0^t r(\tau) h(t-\tau) d\tau}_{\text{Matched Filter}} = \underbrace{\int_0^t r(\tau) s(T - (t-\tau)) d\tau}_{\text{Korrelator}} \overbrace{=}^{t=T} \int_0^T r(\tau) s(\tau) d\tau$$

\section{Informationen}
\begin{itemize}
  \item \textbf{Information} (mathematische Betrachtung): Antwort auf eine konkrete Fragestellung. Die 
  Fragen können geschickter gestellt werden, um die Anzahl der Fragen zu minimieren. Fragen sind dann 
  geschickt gewählt, wenn bei wiederholtem Zufallsexperiment im Schnitt die anzahl Fragen minimal ist.
  Bei binärer Sichtweise sind Fragen dann optimal, wenn sie \textbf{mit 50\% Wahrscheinlichkeit} mit Ja oder
  Nein beantwortet werden. So werden wahrscheinliche Ergebnisse mit wenigen Fragen gefunden.
  \item \textbf{DMS}: Discrete memoryless source (jedes neue Symbol ist unabhängig von vorhergehenden Symbolen).
  \item \textbf{DMC}: Discrete memoryless Channel (der Ausgang $Y_i$ ist nur abhängig von $X_i$, und nicht von $X_k$, $k \neq i$).
\end{itemize}

\subsection{Mathematische Beschreibung}
\begin{tabular}{ll|ll}
  \cdef{$I(X_i)$} & \cdef{Informationsgehalts des Ereignis $X_i$} &
  \cdef{$R(X)$} & \cdef{Redundanz der Quelle X} \\
  \cdef{$H(X)$} & \cdef{Entropie des Alphabets einer Quelle $X$} &
  \cdef{$r$} & \cdef{Symbolrate}\\
  \cdef{$m$} & \cdef{Grösse des Alphabets in anz. Zeichen}&
  \cdef{$R$} & \cdef{Informationsrate}\\
\end{tabular}

\begin{center}
\begin{tabular}{c}
  $I(X_i) = -\log_2{P(X_i)}$ \\
  $I(X_i) \geq 0$ \\
  $I(X_i) > I(X_k), \quad \text{wenn } P(X_i) < P(X_k)$ \\
  $I(X_i, X_k) = I(X_i) + I(X_k), \quad \text{wenn } X_i, X_k \text{ statistisch unabhängig}$ \\
  $\fmm H(X) = \E[I(X_i)] = \sum_i P(X_i) \cdot I(X_i) = - \sum_i P(X_i) \cdot \log_2{ P(X_i)}$ \\
  $0 \leq H(X) \leq \log_2{m} \qquad \text{für binäre Quellen: } 0 \leq H(X) \leq 1$ \\
  $H_{max}(X) = \log_2{m}$ \\
  $\fmm R(X) = H_{max}(X) - H(X) = \log_2{m} + \sum_i P(X_i) \cdot \log_2{P(X_i)}$
  $R = r \cdot H(X)$
\end{tabular}
\end{center}

\subsection{Kanaldarstellung}

Die \cdef{Kanalmatrix} gibt für jedes gesendete Symbol $x_i$ die bedingte Wahrscheinlichkeit 
$P(y_i|y_i)$ an, dass der Empfänger ein $y_i$ dedektiert. 

$$[P(Y|X)] = \left[ \begin{array}{cccc} 
  P(y_1|x_1) & P(y_1|x_1) & \ldots & P(y_n|x_1) \\
  P(y_1|x_2) & P(y_1|x_2) & \ldots & P(y_n|x_2) \\
  \vdots & \vdots & \ddots & \vdots \\
  P(y_1|x_m) & P(y_1|x_m) & \ldots & P(y_n|x_m) \\
\end{array} \right]$$

Die \cdef{Verbundsmatrix} gibt die Wahrscheinlichkeiten von sämtlichen Kombinationen von gesendeten 
und empfangenen Symbolen $X_i$ und $Y_i$ an.

$$[P(Y,X)] = \left[ \begin{array}{cccc} 
  P(y_1,x_1) & P(y_2,x_1) & \ldots & P(y_n,x_1) \\
  P(y_1,x_2) & P(y_2,x_2) & \ldots & P(y_n,x_2) \\
  \vdots & \vdots & \ddots & \vdots \\
  P(y_1,x_m) & P(y_2,x_m) & \ldots & P(y_n,x_m) \\
\end{array} \right]$$

\begin{center}
\begin{tabular}{cc}
  
  \begin{tabular}{l}
    \textbf{Verlustloser Kanal}: \\
    $[P(Y|X)] = \left[ \begin{array}{ccccc} 
      \frac{3}{4} & \frac{1}{4} & 0 & 0 & 0\\
      0 & 0 & \frac{1}{3}& \frac{2}{3} & 0 \\
      0 & 0 & 0 & 0 & 1 \\
    \end{array} \right]$
  \end{tabular} &
  \begin{tabular}{l}
    \begin{tikzpicture}[>=latex, scale=0.6]
      \draw (0,2) node[left] {$x_1$};
      \draw (0,1) node[left] {$x_2$};
      \draw (0,0) node[left] {$x_3$};
      \draw [*-*] (0,2) -- (3,3) node[right] {$y_1$};
      \draw [->] (0,2) -- (1.5,2.5) node[above, yshift=-0.5mm] {\scriptsize $3/4$};
      \draw [-*] (0,2) -- (3,2) node[right] {$y_2$};
      \draw [->] (0,2) -- (1.5,2) node[above right, yshift=-0.5mm] {\scriptsize $1/4$};
      \draw [*-*] (0,1) -- (3,1) node[right] {$y_3$};
      \draw [->] (0,1) -- (1.5,1) node[above, yshift=-0.5mm] {\scriptsize $1/3$};
      \draw [-*] (0,1) -- (3,0) node[right] {$y_4$};
      \draw [->] (0,1) -- (1.5,0.5) node[above right, yshift=-1mm] {\scriptsize $2/3$};
      \draw [*-*] (0,0) -- (3,-1) node[right] {$y_2$};
      \draw [->] (0,0) -- (1.5,-0.5) node[above, yshift=-0.5mm] {\scriptsize $1$};
    \end{tikzpicture}
  \end{tabular} \\
  
  \midrule
  
  \begin{tabular}{l}
    \textbf{Deterministischer Kanal}: \\
    $[P(Y|X)] = \left[ \begin{array}{ccc} 
      1 & 0 & 0 \\
      1 & 0 & 0 \\
      0 & 1 & 0 \\
      0 & 1 & 0 \\
      0 & 0 & 1 \\
    \end{array} \right]$
  \end{tabular} &
  \begin{tabular}{l}
    \begin{tikzpicture}[>=latex, scale=0.6]
      \draw (0,4) node[left] {$x_1$};
      \draw (0,3) node[left] {$x_2$};
      \draw (0,2) node[left] {$x_3$};
      \draw (0,1) node[left] {$x_4$};
      \draw (0,0) node[left] {$x_5$};
      \draw [*-*] (0,4) -- (3,3) node[right] {$y_1$};
      \draw [->] (0,4) -- (1.5,3.5) node[above, yshift=-0.5mm] {\scriptsize $1$};
      \draw [*-] (0,3) -- (3,3);
      \draw [->] (0,3) -- (1.5,3) node[above left, yshift=-0.5mm] {\scriptsize $1$};
      \draw [*-*] (0,2) -- (3,2) node[right] {$y_2$};
      \draw [->] (0,2) -- (1.5,2) node[above, yshift=-0.5mm] {\scriptsize $1$};
      \draw [*-] (0,1) -- (3,2);
      \draw [->] (0,1) -- (1.5,1.5) node[above left, yshift=-1mm] {\scriptsize $1$};
      \draw [*-*] (0,0) -- (3,1) node[right] {$y_2$};
      \draw [->] (0,0) -- (1.5,0.5) node[above, yshift=-0.5mm] {\scriptsize $1$};
    \end{tikzpicture}
  \end{tabular} \\
  
  \midrule
  
  \begin{tabular}{l}
    \textbf{Rauschfreier Kanal}: \\
    $[P(Y|X)] = \left[ \begin{array}{cccc} 
      1 & 0 & 0 & 0\\
      0 & 1 & 0 & 0 \\
      0 & 0 & 1 & 0 \\
      0 & 0 & 0 & 1 \\
    \end{array} \right]$
  \end{tabular} &
  \begin{tabular}{l}
    \begin{tikzpicture}[>=latex, scale=0.6]
      \draw (0,3) node[left] {$x_1$};
      \draw (0,2) node[left] {$x_2$};
      \draw (0,1) node[left] {$x_3$};
      \draw (0,0) node[left] {$x_4$};
      \draw [*-*] (0,3) -- (3,3) node[right] {$y_1$};
      \draw [->] (0,3) -- (1.5,3) node[above, yshift=-0.5mm] {\scriptsize $1$};
      \draw [*-*] (0,2) -- (3,2) node[right] {$y_2$};
      \draw [->] (0,2) -- (1.5,2) node[above, yshift=-0.5mm] {\scriptsize $1$};
      \draw [*-*] (0,1) -- (3,1) node[right] {$y_3$};
      \draw [->] (0,1) -- (1.5,1) node[above, yshift=-0.5mm] {\scriptsize $1$};
      \draw [*-*] (0,0) -- (3,0) node[right] {$y_4$};
      \draw [->] (0,0) -- (1.5,0) node[above, yshift=-0.5mm] {\scriptsize $1$};
    \end{tikzpicture}
  \end{tabular} \\
  \midrule
  
  \begin{tabular}{l}
    \textbf{Binärer symmetrischer Kanal}: \\
    $[P(Y|X)] = \left[ \begin{array}{cc} 
      (1-p) & p \\
      p & (1-p) \\
    \end{array} \right]$
  \end{tabular} &
  
  \begin{tabular}{l}
    \begin{tikzpicture}[>=latex, scale=0.6]
      \draw (0,2) node[left] {$x_1$};
      \draw (0,0) node[left] {$x_2$};
      \draw [*-*] (0,2) -- (3,2) node[right] {$y_1$};
      \draw [->] (0,2) -- (1.5,2) node[below, yshift=   1mm] {\scriptsize$(1-p)$};
      \draw [->] (0,0) -- (1.5,0) node[below, yshift= 0.5mm] {\scriptsize$(1-p)$};
      \draw [*-*] (0,0) -- (3,0) node[right] {$y_2$};
      \draw (0,2) -- (3,0);
      \draw (0,0) -- (3,2);
      \draw [->] (0,2) -- (2,2/3) node[above,right, yshift=0.5mm] {\scriptsize$p$};
      \draw [->] (0,0) -- (1,2/3) node[above,left, yshift=0.5mm] {\scriptsize$p$};
    \end{tikzpicture}
  \end{tabular} \\
\end{tabular}
\end{center}

\subsection{Bedingte Entropie}
\begin{itemize}
  \item \cdef{$H(Y|X)$} ist der Informationsbedarf, um aus gesendetem X das empfangene Y zu bestimmen.
  \item \cdef{$H(X|Y)$} ist der Informationsbedarf um aus empfangenem Y das gesendete X zu bestimmen.
        (Information aus dem Empfänger, welche notwendig ist, um zu wissen, was gesendet wurde.)
  \item \cdef{$H(X,Y)$: die Verbundsentropie} ist die Information des gesamten Kommunikationskanals. 
  Das ist der Informationsgehalt der gesendeten und Empfangenen Symbole, sowie deren Beziehung zueinander.
  \item \cdef{$I(X,Y)$} ist die Information, welche vom Eingang zum Ausgang des Kanals transferiert wird.
\end{itemize}

\begin{center}
  \begin{tikzpicture} [scale=0.8, transform shape]
    \draw (-1,2) -- (-0.5,1.5) node[left, xshift = -2mm]{$H(X)$} -- (-1,1) -- (1,1) -- (0.5,0.5) --
          (0.75,0.5) node[below left]{$H(Y|X)$} -- (0.75,0.25) -- (1.25,0.75) -- (5,0.75) --
          (5.5,1.25) node[right]{$H(Y)$} -- (5,1.75) -- (3,1.75) -- (3.5,2.25) --
          (3.5,2.5) node[above right, yshift=-2mm]{$H(X|Y)$} -- (3.25,2.5) -- (2.75,2) -- (-1,2);

    \draw [>=latex, <->](2,0.75) -- node[right]{$H(X,Y)$} (2,2);
    \draw [dashed] (3,1.75) -- (0,1.75);
    \draw [>=latex, <->](0,1) -- node[right]{$I(X,Y)$} (0,1.75);
  \end{tikzpicture}
\end{center}

$$H(X,Y) = H(X|Y) + H(Y) = H(Y|X) + H(X)$$
$$I(X,Y) = H(X) - H(X|Y) = H(Y) - H(Y|X) = I(Y,X)$$
$$I(X,Y) = H(X) + H(Y) - H(X,Y)$$

\subsection{Kanalkapazität}
Die \cdef{Kanalkapazität $C_s$, $C$} ist die W'keitsverteilung von $X$, welche den Informationsfluss über den 
Kanal maximiert. Dabei ist da Eingangs- und Ausgangsalphabet, sowie die Kanalmatrix $P(Y|X)$ gegeben. Hier ist 
\cdef{$r$: Symbolrate}

$$C = r \cdot C_s = r \cdot \max I(X,Y)$$

\section{Quellenkodierung}
\begin{itemize}
  \item \textbf{Fixed Length Code}: Jedes Codewort hat die gleiche Länge (Bsp: ASCII-Code)
  \item \textbf{Variable Length Code}: Je nach häufigkeit wird die Länge der Codewörter bestummen (Bsp: 
        Morse, Shannon-Fano, Huffmann)
  \item \textbf{Prefix-Free Code}: Kein codewort ist Präfik eines anderen Codeworts (Bsp: Shannon-Fano, Huffman)
  \item \textbf{Uniquely Decodable Code}: Eine Kette von Codewörtern kann eindeutig wieder in die ursprüngliche 
        Symbolfolge zurückgewandelt werden.
  \item \textbf{Instantaneous Code}: Eindeutig decodierbarer Code, welcher nach dem Empfangen jedes einzelnen
        Codeworts sofort ein einduetiges Symbol liefert, ohne dass nachfolgende Symbole decodiert werden müssen.
\end{itemize}

\subsection{Optimaler Code}
Jeder Instantaneous Code, welcher eine minimale Codelänge besitzt, ist ein optimaler Code. 

\begin{tabular}{ll}
  \begin{tabular}{ll}
    \cdef{$L$} & \cdef{Mittlere Codelänge} \\
    \cdef{$\eta$} & \cdef{Effizienz der Codierung} \\
    \cdef{$R$} & \cdef{Redundanz der Codierung} \\
  \end{tabular} &
  \begin{tabular}{l}
    $\fmm L = \sum_i  P(x_i) \cdot n_i \geq H(X)$ \\
    $\fmm \eta = \frac{H(X)}{L}$ \\
    $\fmm R = L - H(X)$
  \end{tabular}
\end{tabular}

\subsection{Kraft'sche Ungleichung}
Gegeben ist eine Quelle mit dem Alphabet ${x_i}, i = {1 \ldots n}$. Jedem Symbol $x_i$ wird eine Codelänge $n_i$ 
zugewiesen. Die Kraft'sche Ungleichung besagt, dass ein eindeutig und sofort codierbarer binärer Code gefunden 
werden kann, wenn:
$$K = \sum_i 2^{-n_i} \leq 1$$

\renewcommand{\arraystretch}{1}

\subsubsection{Shannon-Fano Codierung}
\begin{enumerate}
  \item Symbole nach absteigender W'keit anordnen.
  \item Mit Trennung 2 Teilmengen möglichst gleicher W'keit bilden
  \item Obere Teilmenge 0, unterer Teilmenge 1 zuordnen
  \item für jede Teilmenge Schritt 2 wiederholen.
\end{enumerate}

\subsection{Huffman Codierung}
\begin{enumerate}
  \item Symbole und Symbolgruppen nach absteigender W'keit anordnen.
  \item Unterste zwei Symbole als Symbolgruppe zusammenfassen.
  \item Weiter bei Schritt 1, bis nur noch zwei Symbolgruppen vorliegen.
  \item Der Symbolgruppe mit grösserer W'keit 0, der anderen 1 zuordnen.
  \item Letzte Reduktion rückgängig machen.
  \item Weiter bei Schritt 4, biss für alle Einzelsymbole ein Codewort vorliegt.
\end{enumerate}

\renewcommand{\arraystretch}{1.5}

\section{Kanalcodierung}
Durch das Beifügen von geeigneter Redundanz wird die Fehlererkennund und Fehlerkorrektur
ermöglicht. \cdef{$C_s$ [b / Symbol]: Kanalkapazitätm $H(X)$: Entropie des DMS}.

\textbf{Theorem:} Falls $H(X) < C_s$ kann mit geeigneter Kanalcodierung die Fehlerrate der 
Übertragung beliebig klein gemacht werden. Falls $H(X) > C_s$ist fehlerfreie Übertragung
nicht möglich.

Das heisst, dass Bitfehler in der Übertragung zu einer Beschränkung der nutzbaren 
Übertragungsrate führen.

\begin{ddtabular}
  $\cn{d}, d(x)$ & Datenwort & 
  $\cn{c}, c(x)$ & Codewort \\
  $k$ & Länge des Datenwortes &
  $n$ & Länge des Codewortes \\
  $R_c = \frac{k}{n}$ & Coderate &
  $m$ & Anzahl Symbole im Alphabet \\
  $\cn{s}, s(x)$ & Fehlersyndrom &
  $K = \{ 0 \: 1 \} $ & Zeichen im Alphabet \\
  $\cn{c}_r, c_r(x)$ & Empfangenes Codewort &
  $\cn{e}, e(x)$ & Error-pattern der Länge $n$\\
\end{ddtabular}

\subsection{Blockcodes}

Der Eingangsdatenstrom wird in Blöcke unterteilt, welcher separat codiert wird. 

Ein \textbf{Galois Körper} ist ein algebraischer Kérper mit endlich vielen Elementen. Wir
beschränken uns auf den binären Fall: $M = {0, 1}$. Die Operation $\oplus$ (logisches XOR) und
$\odot$ (logisches AND)

$$\cn{a} = (a_1, a_2, \ldots, a_n) \quad \text{und} \quad \cn{b} = (b_1, b_2, \ldots, b_n) \quad \text{aus dem Code } C$$
$$\cn{c} = \cn{a} \oplus \cn{b} = (a_1 \oplus b_1, a_2 \oplus b_2, \ldots, a_n \oplus b_n)$$

\subsection{Linearer Code}

Ein Code $C$ ist genau dann linear, wenn für alle $\cn{a}$ und $\cn{b}$ auch $\cn{c} = \cn{a} \oplus \cn{b}$
zu $C$ gehört. 

\begin{dtabular}
  $w(\cn{c})$ & Hamming-Gewicht: anzahl Einer des Codewortes \\
  $d(\cn{a}, \cn{b})$ & Hamming-Distanz : Anzahl unterschiedliche Stellen von $\cn{a}$ und $\cn{b}$\\
  $d_{min}$ & Minimale Hamming-Distanz aller möglichen Codewort-paare \\
\end{dtabular}

$$w(\cn{c}) = d(\cn{c}, \cn{0}) \qquad d(\cn{a}, \cn{b}) = w(\cn{a} \oplus \cn{b})$$
$$d_{min} = \min [ d(\cn{a}, \cn{b})] \stackrel{\text{linear}}{=} \min [ w(\cn{a} \oplus \cn{b})] = \min [w(\cn{c})]$$

Anhand dieser minimalen Hammingdistanz $d_{min}$ lässt sich die Anzahl der korrigierbaren und detektierbaren Fehler berechnen:

\begin{tabular}{ll}
  \begin{dtabular}
    $t_d$ & Detektierbare Fehler \\
    $t_c$ & Korrigierbare Fehler \\
  \end{dtabular} &
  \begin{mtabular}{l}
    $t_d = d_{min} - 1$ \\
    $\fmm t_c=\frac{1}{2} \cdot (d_{min} - 1) $
  \end{mtabular}
\end{tabular}

Bei \textbf{Systematischen Blockcodes} kommen alle Datenbits unmodifiziert im Codewort vor. 
Falls diese Bits an einer definierten Stelle auftreten, handelt es sich um ein \textbf{geordneter systematischer Code}.

\subsubsection{Generatormatrix $G$}
\cdef{$I_k$ ist die Einheitsmatrix} und \cdef{$P^T$ die transponierte Paritätsmatrix}.

$$\cn{c} = \cn{d} \odot G = \cn{d} \odot \left[ I_K \quad P^T \right]$$
$$\cn{c} = \left[ \begin{array}{llll} d_1 & d_2 & \cdots & d_k \end{array} \right] \odot 
  \left[ \begin{array}{llllllll} 
    1 & 0 & \cdots & 0 & p_{11} & p_{21} & \cdots & p_{m1} \\
    0 & 1 & \cdots & 0 & p_{12} & p_{22} & \cdots & p_{m2} \\
    \vdots & \vdots & \ddots & \vdots & \vdots & \vdots & \ddots & \vdots \\
    0 & 0 & \cdots & 1 & p_{1k} & p_{2k} & \cdots & p_{mk} \\
  \end{array} \right]
$$

\subsubsection{Praitätsprüfmatrix $H$}
Mit dieser Matrix können Übertragungsfehler erkannt werden. $O$ ist die $[k \times m]$ 
Nullmatrix, $\cn{0}$ ist der $[1 \times m]$ Nullvektor

$$H = \left[ \begin{array}{ll} P & I_m \end{array} \right] \qquad H^T = 
\left[ \begin{array}{l} P^T \\ I_m \end{array} \right]$$
$$G \odot H^T = O \qquad \cn{c} \odot H^T = \cn{d} \odot G \odot H^T = \cn{0}$$
$$\text{Wenn:} \: \cn{c} \odot H^T = 0 \: \rightarrow \: \text{keine Fehler bei der Übertragung}$$

Die minimale Hamming-Distanz $d_{min}$ von $C$ entspricht somit der minimalen Anzahl Zeilen 
von $H^T$, welche linear kombiniert den Nullvektor $\cn{0}$ ergeben. 
\cdef{$\cn{c}_r$ ist das empfangene Codewort}, welches maximal \cdef{$t$ Fehler} haben kann, um das Codewort korrigieren zu können.

$$\cn{c_r} = \cn{c} \oplus \cn{e}$$ 
$$\cn{s} = \cn{c_r} \odot H^T = (\cn{c} \oplus \cn{e}) \odot H^T = \cn{c} \odot H^T \oplus \cn{e} \odot H^T = \cn{e} \odot H^T$$

Bei Einzelfehler entspricht das Fehlersyndrom $\cn{s}$ genau einer Zeile von $H^T$. 
Wenn alle Zeilen von $H^T$ unterschiedlich sind, kann aus $\cn{s}$ das Fehlerbit $e_i$ eindeutig bestimmt werden.
So kann das empfangene Codewort $\cn{c_r}$ korrigiert werden. 

$$2^{n-k} \geq \sum_{i=0}^t \binom{n}{i}$$

Gilt das Gleichheitszeichen, handelt es sich um einen so genannten perfekten Code. 
Ein perfekter Code, der nur Einzelfehler korrigieren kann, nennt man \cdef{Hamming-Code}.

\subsection{Zyklische Lineare Codes}
Jede Zyklische Verschiebung eines Codewortes $\cn{c}$ zu einem neuen gültigen Codewort von $C$ führt.
Um hohe Coderate zu erzielen, sollte eine möglichst hohe Coderate $R_c = \frac{k}{n}$ erreicht werden
Ein Codewort $\cn{c}$ kann als Polynom $c(x)$ geschrieben werden:

$$\cn{c} = (c_0, \, c_1, \, c_2, \, \ldots \, c_{n-1}) \qquad c(x) = c_0 + c_1 x + c_2 x^2 + \ldots + c_{n-1} x^{n-1}$$

Es werden folgende Operationen definiert:

$$\mod_2(x^k + x^k) = x^k \oplus x^k = 0 \qquad \mod_2 (x^k - x^k) = x^k \ominus x^k = 0$$
$$\mod_2(0+x^k) = 0 \oplus x^k = x^k \qquad \mod_2(0-x^k) = 0 - x^k = x^k$$

$$\text{Beispiele:} \: 
\begin{array}{ll}
  \cn{a} = (1\,1\,0\,0\,0) & a(x) = 1 + x + x^2 \\
  \cn{b} = (1\,0\,1\,1\,0) & b(x) = 1 + x^2 + x^3 \\
\end{array}$$
$$\cn{a} \odot \cn{b} = a(x) \cdot b(x) = 1 + x + 2x^2 + 2x^3 + 2x^4 + x^5 = 1 + x + x^5 \Rightarrow (1\,1\,0\,0\,1)$$

Im Folgenden wird eine wichtige Operation beschrieben. 
Für das Polynom $q(x)$ interessiert man sich meisst nicht.
Normalerweise wird nach dem Rest $r(x)$ gesucht. $r(x)$ und $q(x)$.
Die Operation $\oslash$ ist eine Polynomdivision.

$$f(x) = q(x) \cdot h(x) + r(x) \qquad f(x) \oslash h(x) \: \Rightarrow \: q(x), \, r(x) $$
$$\text{Notation:} \qquad r(x) = f(x) \: \mod h(x) $$

Wenn ein Codewort $c$ um $i$ Stellen zyklisch verschoben wird: 
$$\cn{c}^{(i)} = (c_{n-i},\,c_{n-(i-1)},\,\ldots,\,c_{n-1},\,c_0,\,c_1,\,\ldots,\,c_{n-(i+1)})$$
$$c^{(i)}(x) = c_{n-i} + c_{n-(i-1)} x + \ldots + c_{n-1} x^{i-1} + c_0 x^i + c_1 x^{i-1} + \ldots + c_{n-(i+1)}x^{n-1}$$
$$c^{(i)}(x) = x^i c(x) \: \mod \left(1+x^n\right)$$

\subsubsection{Fundamentales Theorem für zyklische Codes}
Gegeben sei das \cdef{Generatorpolynom $g(x)$} mit der Ordnung $(n-k)$. 
Jeder $(n,\,k)$ zyklische Code $C$ kann mit Hilfe eines solchen Polynoms $g(x)$ gebildet werden:

$$\text{\textbf{Bedingungen}:} \qquad g_{n-k} = g_0 = 1 \qquad (x^n + 1) = q(x) \odot g(x) \quad \text{(ohne Rest)}$$
$$c(x) = d(x) \odot g(x)$$

\subsubsection{Fehlersyndrom}
Im empfangenen Codewort $c_r(x)$ muss der Fehler $e(x)$ gefunden werden. 
Das Fehlersyndrom $s(x)$ ist ein Polynom des Grads $n-k$.

$$c_r(x) = c(x) \oplus e(x)$$
$$s(x) = r(x) \: \mod g(x)$$
$$s(x) = e(x) \: \mod g(x)$$


\subsubsection{Generatormatrix}
$$G = \left(\begin{array}{c} g(x) \\ x \cdot g(x) \\ x^2 \cdot g(x) \\ \vdots \\ x^{k-1} \cdot g(x) \end{array}\right)$$


\subsubsection{Hardware-Realisierung}
\begin{dtabular}
  Generatorpolynom & $g(x) = g_0 + g_1 x + g_2 x^2 + \ldots + g_{n-k} x^{n-k}$ \\
  Datenwort & $d(x) = d_0 + d_1 x + d_2 x^2 + \ldots + d_{k-1} x^{k-1}$ \\
  Parity Bits & $p(x) = p_0 + p_1 x + p_2 x^2 + \ldots + p_{n-k-1} x^{n-k-1}$ \\
  Codewort & $c(x) = p_0 + \ldots + p_{n-k-1} x^{n-k-1} + d_0 x^{n-k} + \ldots + d_{k-1} x^{n-1}$ \\
\end{dtabular}

$$c(x) = q(x) \cdot g(x) = p(x) + x^{n-k} \cdot d(x)$$
$$x^{n-k} \cdot d(x) = q(x) \cdot g(x) + p(x)$$


\begin{tikzpicture}[auto, scale=0.6, transform shape, node distance=1.5cm]
  %help lines
  %\draw [draw=clGray, help lines, step=.1] (-1,-1) grid (10,2);
  %\draw [draw=clGray, thick, help lines, step=1] (-1,-3) grid (18,4);
  
  %place nodes
  \draw
    node at (0,0) [block] (z1) {\Large $z^{-1}$}
    node [sum, right of=z1] (sum1) {\large $+$}
    node [block, right of=sum1] (z2) {\Large $z^{-1}$}
    node [sum, right of=z2] (sum2) {\large $+$}
    node [block, right of=sum2] (z3) {\Large $z^{-1}$}
    node [sum, right of=z3] (sum3) {\large $+$}
    node [block, right of=sum3] (z4) {\Large $z^{-1}$}
    node at (11,0) (dots1) {\Large \ldots}
    node at (11,3) (dots2) {\Large \ldots}
    node at (13,0) [sum] (sum4) {\large $+$}
    node [block, right of=sum4] (z5) {\Large $z^{-1}$} 
    node at (17,0) [sum] (sum5) {\large $+$}
    node at (-1,1.5) [block] (g0) {\Large $g_0$}
    node [block, above of=sum1] (g1) {\Large $g_1$}
    node [block, above of=sum2] (g2) {\Large $g_2$}
    node [block, above of=sum3] (g3) {\Large $g_3$}
    node [block, above of=sum4] (g4) {\Large $g_{n-k-1}$};
    %node at (15,0) [] (input) {$d(x)$};
    
  %draw connectors
  \draw [->] (g0)   |- (z1);
  \draw [->] (z1)   -- (sum1);
  \draw [->] (sum1) -- (z2);
  \draw [->] (z2)   -- (sum2);
  \draw [->] (sum2) -- (z3);
  \draw [->] (z3)   -- (sum3);
  \draw [->] (sum3) -- (z4);
  \draw      (z4)   -- (dots1); 
  \draw [->] (g1)   -- (sum1);
  \draw [->] (g2)   -- (sum2);
  \draw [->] (g3)   -- (sum3);
  \draw [->] (dots2)-| (g3);
  \draw [->] (dots2)-| (g2);
  \draw [->] (dots2)-| (g1);
  \draw [->] (dots2)-| (g0);
  
  \draw [->] (sum5) |- (15,3);
  \draw [fill=black] (15,3) circle (1mm);
  \draw [cRed, thick]  (15,3) -- ++ (150:1);
  \draw [cBlue, thick] (15,3) -- ++ (175:1);
  \draw [fill=black] (14,3) circle (1mm);
  \draw [->] (14,3) -| (g4);
  \draw (14,3) -- (dots2);
  \draw [->] (dots1)-- (sum4);
  \draw [->] (sum4) -- (z5);
  \draw [->] (z5)   -- (sum5);
  \draw [->] (g4)   -- (sum4);
  
  \draw [fill=black] (17,-2)   circle (.5mm) 
                     (18,-2)   circle (1mm)
                     (15.75,0) circle (.5mm)
                     (18,-1)   circle (1mm)
                     (19,-1.5) circle (1mm);
  \draw (15,-2) node [left] {$d(x)$} -- (18,-2);
  \draw [->] (17,-2) -- (sum5);
  \draw (15.75,0) |- (18,-1);
  \draw [->] (19,-1.5) -- (20,-1.5) node [right] {$c(x)$};
  \draw [thick, cRed]  (19,-1.5) -- ++(155:1.15);
  \draw [thick, cBlue] (19,-1.5) -- ++(205:1.15);
\end{tikzpicture}

\begin{itemize}
  \item \textbf{\color{cBlue} Phase 1:} Ausgabe von $d(x)$. ($k$ Zyklen)
  \item \textbf{\color{cRed}  Phase 2:} Ausgabe von $p(x) = $ Rest. ($n-k$ Zyklen)
\end{itemize}

\section{Anhang}

\subsection{Fourier Transformation}
$$X(\omega) = \fourier(x(t)) = \int_{-\infty}^\infty x(t) \cdot e^{-j \omega t} \cdogt dt $$
$$x(t) = \fourier^{-1}(X(\omega)) = \frac{1}{2\pi} \int_{-\infty}^\infty X(\omega) \cdot e^{j \omega t} \cdot d\omega$$

\subsubsection{Eigenschaften}
\begin{tabular}{lcc}
  Eigenschaft & $x(t)$ & $X(\omega)$ \\
  \toprule
  Linearität & $a_1 x_1(t) + a_2 x_2(t)$ & $a_1 X_1(\omega) + a_2 X_2(\omega)$ \\
  Zeitverschiebung & $x(t-t_0)$ & $X(\omega) \cdot e^{-j \omega t_0}$ \\
  Zeitskalierung & $x(at)$ & $\frac{1}{\abs{a}} \cdot X\left(\frac{\omega}{a}\right)$ \\
  Dualität & $X(t)$ & $2 \pi x(-\omega)$ \\
  Frequenz verschiebung & $x(t) \cdot e^{j \omega_0 t}$ & $X(\omega - \omega_0)$ \\
  Modulation & $x(t) \cdot \cos \omega_0 t$ & $\frac{1}{2} \left[ X(\omega - \omega_0) + X(\omega + \omega_0) \right]$ \\
  Ableitung nach $t$ & $\dot{x}(t)$ & $j \omega X(\omega)$ \\
  Ableitung nach $\omega$ & $-j t x(t)$ & $X'(\omega)$ \\
  Integration & $\int_{-\infty}^t x(\tau) d\tau$ & $\frac{1}{j\omega} X(\omega) + \pi X(\omega) \delta(\omega)$  \\
  Convolution & $x_1(t) \ast x_2(t)$ & $X_1(\omega) \cdot X_2(\omega)$ \\
  Multiplication & $x_1(t) \cdot x_2(t)$ & $\frac{1}{2\pi} X_1(\omega) \ast X_2(\omega)$ 
\end{tabular}

\subsubsection{Einige Transformationspaare}

\begin{tabular}{cc}
  $x(t)$ & $X(\omega)$ \\
  \toprule 
  $\delta(t)$ & $1$ \\
  $\delta(t-t_0)$ & $e^{j \omega t_0}$ \\
  $1$ & $2 \pi \delta(\omega)$ \\
  $u(t) = \left\{ \begin{array}{ll} 1 & \text{falls } t > 0 \\ 0 & \text{sonst} \end{array} \right.$ & $\fmm \pi \delta(\omega) + \frac{1}{j\omega}$ \\
  $\sgn(t)$ & $\fmm \frac{2}{j\omega}$ \\
  $\fmm \frac{1}{\pi t}$ & $-j \cdot \sgn(\omega)$ \\
  $e^{j \omega_0 t}$ & $2 \pi \delta(\omega - \omega_0)$ \\
  $\cos \omega_0 t$ & $\pi \left[ \delta(\omega - \omega_0) + \delta(\omega + \omega_0) \right]$ \\
  $\sin \omega_0 t$ & $-j \pi \left[ \delta(\omega - \omega_0) - \delta(\omega + \omega_0) \right]$ \\
  $e^{-a t} \cdot u(t), \quad a > 0$ & $\fmm \frac{1}{j \omega + 1}$ \\
  $t e^{-a t} \cdot u(t), \quad a > 0$ & $\fmm \frac{1}{\left( j \omega + 1 \right)^2}$ \\
  $\fmm e^{-t^2 / 2\sigma^2 }$ & $\sigma \sqrt{2\pi} \cdot e^{-\sigma^2 \omega^2 / 2}$ \\
  $p_a(t) = \left\{ \begin{array}{ll} 1 & \text{falls } \abs{t} < a \\ 0 & \text{sonst} \end{array} \right.$ & $\fmm 2a \cdot \frac{\sin \omega a}{\omega a}$ \\
  $\fmm \frac{\sin at}{\pi t}$ & $p_a(\omega) = \left\{ \begin{array}{ll} 1 & \text{falls } \abs{\omega} < a \\ 0 & \text{sonst} \end{array} \right.$ \\
  $\Lambda_a(t) = \left\{ \begin{array}{ll} 1 - \frac{\abs{t}}{a} & \text{falls } \abs{t} < a \\ 0 & \text{sonst} \end{array} \right.$ & $\fmm a \left[ \frac{\sin \left(\frac{\omega a}{2}\right)}{\frac{\omega a}{2}} \right]^2$ \\
  $\fmm \sum_{n = -\infty}^{\infty} \delta(t - nT)$ & $\fmm \frac{2\pi}{T} \sum_{n = -\infty}^{\infty} \delta(\omega - n \frac{2\pi}{T})$ \\
  $\fmm \hat{x}(t) = \frac{1}{\pi} \int_{-\infty}^\infty \frac{x(\tau)}{t - \tau}d \tau$ & $-j \sgn(\omega) \cdot X(\omega)$ \\
\end{tabular}

% \begin{tabular}{ll}
%   \textbf{Variante 1: LSB first} & \textbf{Variante 2: MSB first} \\
%   \begin{tabular}{l}
%     \begin{tikzpicture} [scale=0.5, transform shape]
%       %place nodes
%       \draw
%         node at (0,0) [input, name=input]{}
%         node at (3,0) [block] (z1) {\Large $z^{-1}$}
%         node at (6,0) [block] (z2) {\Large $z^{-1}$}
%         node at (9,0) [block] (z3) {\Large $z^{-1}$}
%         node at (1.5 ,1.5) [block] (g0) {\Large $g_0$}
%         node at (4.5 ,1.5) [block] (g1) {\Large $g_1$}
%         node at (7.5 ,1.5) [block] (g2) {\Large $g_2$}
%         node at (10.5,1.5) [block] (g3) {\Large $g_3$}
%         node at (4.5 ,3) [sum](sum1) {\Large $+$}
%         node at (7.5 ,3) [sum](sum2) {\Large $+$}
%         node at (10.5,3) [sum](sum3) {\Large $+$};
%         
%       \draw      (input)  -- node [above] {\Large $d(x)$} (1.5,0);
%       \draw [fill=black] (1.5,0) circle (1mm);
%       \draw [->] (1.5,0)  -- (z1);
%       \draw [->] (1.5,0)  -- (g0);
%       \draw      (g0)     -- (1.5,3);
%       \draw [->] (1.5,3)  -- (sum1);
%       \draw [->] (sum1)   -- (sum2);
%       \draw [->] (sum2)   -- (sum3);  
%       \draw [->] (z1)     -- (z2);  
%       \draw [->] (4.5,0)  -- (g1);
%       \draw [fill=black] (4.5,0) circle (1mm);
%       \draw [->] (g1)     -- (sum1);
%       \draw [->] (z2)     -- (z3);  
%       \draw [->] (7.5,0)  -- (g2);
%       \draw [fill=black] (7.5,0) circle (1mm);
%       \draw [->] (g2)     -- (sum2);
%       \draw      (z3)     -- (10.5,0);
%       \draw [->] (10.5,0) -- (g3);
%       \draw [->] (g3)     -- (sum3);
%       \draw [->] (sum3) -- node [above] {\Large $c(x)$}(12,3);
%     \end{tikzpicture}
%   \end{tabular} &
%   \begin{tabular}{l}\begin{tikzpicture} [scale=0.5, transform shape]
%       %place nodes
%       \draw
%         node at (0,0) [input, name=input]{}
%         node at (3,0) [block] (z1) {\Large $z^{-1}$}
%         node at (6,0) [block] (z2) {\Large $z^{-1}$}
%         node at (9,0) [block] (z3) {\Large $z^{-1}$}
%         node at (1.5 ,1.5) [block] (g0) {\Large $g_3$}
%         node at (4.5 ,1.5) [block] (g1) {\Large $g_2$}
%         node at (7.5 ,1.5) [block] (g2) {\Large $g_1$}
%         node at (10.5,1.5) [block] (g3) {\Large $g_0$}
%         node at (4.5 ,3) [sum](sum1) {\Large $+$}
%         node at (7.5 ,3) [sum](sum2) {\Large $+$}
%         node at (10.5,3) [sum](sum3) {\Large $+$};
%         
%       \draw      (input)  -- node [above] {\Large $d(x)$} (1.5,0);
%       \draw [fill=black] (1.5,0) circle (1mm);
%       \draw [->] (1.5,0)  -- (z1);
%       \draw [->] (1.5,0)  -- (g0);
%       \draw      (g0)     -- (1.5,3);
%       \draw [->] (1.5,3)  -- (sum1);
%       \draw [->] (sum1)   -- (sum2);
%       \draw [->] (sum2)   -- (sum3);  
%       \draw [->] (z1)     -- (z2);  
%       \draw [->] (4.5,0)  -- (g1);
%       \draw [fill=black] (4.5,0) circle (1mm);
%       \draw [->] (g1)     -- (sum1);
%       \draw [->] (z2)     -- (z3);  
%       \draw [->] (7.5,0)  -- (g2);
%       \draw [fill=black] (7.5,0) circle (1mm);
%       \draw [->] (g2)     -- (sum2);
%       \draw      (z3)     -- (10.5,0);
%       \draw [->] (10.5,0) -- (g3);
%       \draw [->] (g3)     -- (sum3);
%       \draw [->] (sum3) -- node [above] {\Large $c(x)$}(12,3);
%     \end{tikzpicture}
%   \end{tabular}
% \end{tabular}

\end{twocolumn}
\end{document}
